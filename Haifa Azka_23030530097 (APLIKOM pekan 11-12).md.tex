% Options for packages loaded elsewhere
\PassOptionsToPackage{unicode}{hyperref}
\PassOptionsToPackage{hyphens}{url}
\documentclass[
]{book}
\usepackage{xcolor}
\usepackage{amsmath,amssymb}
\setcounter{secnumdepth}{-\maxdimen} % remove section numbering
\usepackage{iftex}
\ifPDFTeX
  \usepackage[T1]{fontenc}
  \usepackage[utf8]{inputenc}
  \usepackage{textcomp} % provide euro and other symbols
\else % if luatex or xetex
  \usepackage{unicode-math} % this also loads fontspec
  \defaultfontfeatures{Scale=MatchLowercase}
  \defaultfontfeatures[\rmfamily]{Ligatures=TeX,Scale=1}
\fi
\usepackage{lmodern}
\ifPDFTeX\else
  % xetex/luatex font selection
\fi
% Use upquote if available, for straight quotes in verbatim environments
\IfFileExists{upquote.sty}{\usepackage{upquote}}{}
\IfFileExists{microtype.sty}{% use microtype if available
  \usepackage[]{microtype}
  \UseMicrotypeSet[protrusion]{basicmath} % disable protrusion for tt fonts
}{}
\makeatletter
\@ifundefined{KOMAClassName}{% if non-KOMA class
  \IfFileExists{parskip.sty}{%
    \usepackage{parskip}
  }{% else
    \setlength{\parindent}{0pt}
    \setlength{\parskip}{6pt plus 2pt minus 1pt}}
}{% if KOMA class
  \KOMAoptions{parskip=half}}
\makeatother
\usepackage{graphicx}
\makeatletter
\newsavebox\pandoc@box
\newcommand*\pandocbounded[1]{% scales image to fit in text height/width
  \sbox\pandoc@box{#1}%
  \Gscale@div\@tempa{\textheight}{\dimexpr\ht\pandoc@box+\dp\pandoc@box\relax}%
  \Gscale@div\@tempb{\linewidth}{\wd\pandoc@box}%
  \ifdim\@tempb\p@<\@tempa\p@\let\@tempa\@tempb\fi% select the smaller of both
  \ifdim\@tempa\p@<\p@\scalebox{\@tempa}{\usebox\pandoc@box}%
  \else\usebox{\pandoc@box}%
  \fi%
}
% Set default figure placement to htbp
\def\fps@figure{htbp}
\makeatother
\setlength{\emergencystretch}{3em} % prevent overfull lines
\providecommand{\tightlist}{%
  \setlength{\itemsep}{0pt}\setlength{\parskip}{0pt}}
\usepackage{bookmark}
\IfFileExists{xurl.sty}{\usepackage{xurl}}{} % add URL line breaks if available
\urlstyle{same}
\hypersetup{
  hidelinks,
  pdfcreator={LaTeX via pandoc}}

\author{}
\date{}

\begin{document}
\frontmatter

\mainmatter
\chapter{Visualisasi dan Perhitungan Geometri dengan EMT}\label{visualisasi-dan-perhitungan-geometri-dengan-emt}

Euler menyediakan beberapa fungsi untuk melakukan visualisasi dan perhitungan geometri, baik secara numerik maupun analitik (seperti biasanya tentunya, menggunakan Maxima). Fungsi-fungsi untuk visualisasi dan perhitungan geometeri tersebut disimpan di dalam file program ``geometry.e'', sehingga file tersebut harus dipanggil sebelum menggunakan fungsi-fungsi atau perintah-perintah untuk geometri.

\textgreater load geometry

\begin{verbatim}
Numerical and symbolic geometry.
\end{verbatim}

\section{Fungsi-fungsi Geometri}\label{fungsi-fungsi-geometri}

Fungsi-fungsi untuk Menggambar Objek Geometri:

defaultd:=textheight()*1.5: nilai asli untuk parameter d\\
setPlotrange(x1,x2,y1,y2): menentukan rentang x dan y pada bidang

koordinat

setPlotRange(r): pusat bidang koordinat (0,0) dan batas-batas sumbu-x dan y adalah -r sd r

plotPoint (P, ``P''): menggambar titik P dan diberi label ``P''

plotSegment (A,B, ``AB'', d): menggambar ruas garis AB, diberi label ``AB'' sejauh d

plotLine (g, ``g'', d): menggambar garis g diberi label ``g'' sejauh d

plotCircle (c,``c'',v,d): Menggambar lingkaran c dan diberi label ``c''

plotLabel (label, P, V, d): menuliskan label pada posisi P

Fungsi-fungsi Geometri Analitik (numerik maupun simbolik):

turn(v, phi): memutar vektor v sejauh phi\\
turnLeft(v): memutar vektor v ke kiri\\
turnRight(v): memutar vektor v ke kanan\\
normalize(v): normal vektor v\\
crossProduct(v, w): hasil kali silang vektorv dan w.\\
lineThrough(A, B): garis melalui A dan B, hasilnya {[}a,b,c{]} sdh.

ax+by=c.

lineWithDirection(A,v): garis melalui A searah vektor v

getLineDirection(g): vektor arah (gradien) garis g

getNormal(g): vektor normal (tegak lurus) garis g

getPointOnLine(g): titik pada garis g

perpendicular(A, g): garis melalui A tegak lurus garis g

parallel (A, g): garis melalui A sejajar garis g

lineIntersection(g, h): titik potong garis g dan h

projectToLine(A, g): proyeksi titik A pada garis g

distance(A, B): jarak titik A dan B

distanceSquared(A, B): kuadrat jarak A dan B

quadrance(A, B): kuadrat jarak A dan B

areaTriangle(A, B, C): luas segitiga ABC

computeAngle(A, B, C): besar sudut \textless ABC

angleBisector(A, B, C): garis bagi sudut \textless ABC

circleWithCenter (A, r): lingkaran dengan pusat A dan jari-jari r

getCircleCenter(c): pusat lingkaran c

getCircleRadius(c): jari-jari lingkaran c

circleThrough(A,B,C): lingkaran melalui A, B, C

middlePerpendicular(A, B): titik tengah AB

lineCircleIntersections(g, c): titik potong garis g dan lingkran c

circleCircleIntersections (c1, c2): titik potong lingkaran c1 dan c2

planeThrough(A, B, C): bidang melalui titik A, B, C

Fungsi-fungsi Khusus Untuk Geometri Simbolik:

getLineEquation (g,x,y): persamaan garis g dinyatakan dalam x dan y\\
getHesseForm (g,x,y,A): bentuk Hesse garis g dinyatakan dalam x dan

y dengan titik A pada

sisi positif (kanan/atas) garis

quad(A,B): kuadrat jarak AB

spread(a,b,c): Spread segitiga dengan panjang sisi-sisi a,b,c, yakni sin(alpha)\^{}2 dengan

alpha sudut yang menghadap sisi a.

crosslaw(a,b,c,sa): persamaan 3 quads dan 1 spread pada segitiga dengan panjang sisi a, b, c.

triplespread(sa,sb,sc): persamaan 3 spread sa,sb,sc yang memebntuk suatu segitiga

doublespread(sa): Spread sudut rangkap Spread 2*phi, dengan sa=sin(phi)\^{}2 spread a.

\section{Contoh 1: Luas, Lingkaran Luar, Lingkaran Dalam Segitiga}\label{contoh-1-luas-lingkaran-luar-lingkaran-dalam-segitiga}

Untuk menggambar objek-objek geometri, langkah pertama adalah menentukan rentang sumbu-sumbu koordinat. Semua objek geometri akan digambar pada satu bidang koordinat, sampai didefinisikan bidang koordinat yang baru.

\textgreater setPlotRange(-0.5,2.5,-0.5,2.5); // mendefinisikan bidang koordinat baru

Sekarang tetapkan tiga titik dan gambarkan.

\textgreater A={[}1,0{]}; plotPoint(A,``A''); // definisi dan gambar tiga titik

\textgreater B={[}0,1{]}; plotPoint(B,``B'');

\textgreater C={[}2,2{]}; plotPoint(C,``C'');

Lalu tiga segmen.

\textgreater plotSegment(A,B,``c''); // c=AB

\textgreater plotSegment(B,C,``a''); // a=BC

\textgreater plotSegment(A,C,``b''); // b=AC

Fungsi geometri meliputi fungsi untuk membuat garis dan lingkaran. Format garisnya adalah {[}a,b,c{]} yang mewakili garis dengan persamaan ax+by=c.

\textgreater lineThrough(B,C) // garis yang melalui B dan C

\begin{verbatim}
[-1,  2,  2]
\end{verbatim}

Hitung garis tegak lurus yang melalui A di BC.

\textgreater h=perpendicular(A,lineThrough(B,C)); // garis h tegak lurus BC melalui A

Dan persimpangannya dengan BC.

\textgreater D=lineIntersection(h,lineThrough(B,C)); // D adalah titik potong h dan BC

Plotkan itu.

\textgreater plotPoint(D,value=1); // koordinat D ditampilkan

\textgreater aspect(1); plotSegment(A,D): // tampilkan semua gambar hasil plot\ldots()

\begin{figure}
\centering
\pandocbounded{\includegraphics[keepaspectratio]{images/Haifa Azka_23030530097 (APLIKOM pekan 11-12)-001.png}}
\caption{images/Haifa\%20Azka\_23030530097\%20(APLIKOM\%20pekan\%2011-12)-001.png}
\end{figure}

Hitung luas ABC:

\[L_{\triangle ABC}= \frac{1}{2}AD.BC.\]\textgreater norm(A-D)*norm(B-C)/2 // AD=norm(A-D), BC=norm(B-C)

\begin{verbatim}
1.5
\end{verbatim}

Bandingkan dengan rumus determinan.

\textgreater areaTriangle(A,B,C) // hitung luas segitiga langusng dengan fungsi

\begin{verbatim}
1.5
\end{verbatim}

Cara lain menghitung luas segitigas ABC:

\textgreater distance(A,D)*distance(B,C)/2

\begin{verbatim}
1.5
\end{verbatim}

Sudut di C.

\textgreater degprint(computeAngle(B,C,A))

\begin{verbatim}
36°52'11.63''
\end{verbatim}

Sekarang lingkaran luar segitiga.

\textgreater c=circleThrough(A,B,C); // lingkaran luar segitiga ABC

\textgreater R=getCircleRadius(c); // jari2 lingkaran luar

\textgreater O=getCircleCenter(c); // titik pusat lingkaran c

\textgreater plotPoint(O,``O''); // gambar titik ``O''

\textgreater plotCircle(c,``Lingkaran luar segitiga ABC''):

\begin{figure}
\centering
\pandocbounded{\includegraphics[keepaspectratio]{images/Haifa Azka_23030530097 (APLIKOM pekan 11-12)-003.png}}
\caption{images/Haifa\%20Azka\_23030530097\%20(APLIKOM\%20pekan\%2011-12)-003.png}
\end{figure}

Tampilkan koordinat titik pusat dan jari-jari lingkaran luar.

\textgreater O, R

\begin{verbatim}
[1.16667,  1.16667]
1.17851130198
\end{verbatim}

Sekarang akan digambar lingkaran dalam segitiga ABC. Titik pusat lingkaran dalam adalah titik potong garis-garis bagi sudut.

\textgreater l=angleBisector(A,C,B); // garis bagi \textless ACB

\textgreater g=angleBisector(C,A,B); // garis bagi \textless CAB

\textgreater P=lineIntersection(l,g) // titik potong kedua garis bagi sudut

\begin{verbatim}
[0.86038,  0.86038]
\end{verbatim}

Tambahkan semuanya ke plot.

\textgreater color(5); plotLine(l); plotLine(g); color(1); // gambar kedua garis bagi sudut

\textgreater plotPoint(P,``P''); // gambar titik potongnya

\textgreater r=norm(P-projectToLine(P,lineThrough(A,B))) // jari-jari lingkaran dalam

\begin{verbatim}
0.509653732104
\end{verbatim}

\textgreater plotCircle(circleWithCenter(P,r),``Lingkaran dalam segitiga ABC''): // gambar lingkaran dalam

\begin{figure}
\centering
\pandocbounded{\includegraphics[keepaspectratio]{images/Haifa Azka_23030530097 (APLIKOM pekan 11-12)-004.png}}
\caption{images/Haifa\%20Azka\_23030530097\%20(APLIKOM\%20pekan\%2011-12)-004.png}
\end{figure}

\chapter{LATIHAN}\label{latihan}

\begin{enumerate}
\def\labelenumi{\arabic{enumi}.}
\item
  Tentukan ketiga titik singgung lingkaran dalam dengan sisi-sisi segitiga ABC.
\item
  Gambar segitiga dengan titik-titik sudut ketiga titik singgung tersebut. Merupakan segitiga apakah itu?
\end{enumerate}

\textgreater setPlotRange(-0.5,5,-0.5,5);

\textgreater A={[}0,1{]}; plotPoint(A,``A'');

\textgreater B={[}2,3{]}; plotPoint(B,``B'');

\textgreater C={[}3,1{]}; plotPoint(C,``C'');

\textgreater plotSegment(A,B,``c''); // c=AB

\textgreater plotSegment(B,C,``a''); // a=BC

\textgreater plotSegment(A,C,``b''); // b=AC

\textgreater c=circleThrough(A,B,C);

\textgreater R=getCircleRadius(c);

\textgreater O=getCircleCenter(c);

\textgreater plotPoint(O,``O'')

\textgreater plotCircle(c,``Lingkaran luar segitiga ABC''):

\begin{figure}
\centering
\pandocbounded{\includegraphics[keepaspectratio]{images/Haifa Azka_23030530097 (APLIKOM pekan 11-12)-005.png}}
\caption{images/Haifa\%20Azka\_23030530097\%20(APLIKOM\%20pekan\%2011-12)-005.png}
\end{figure}

\begin{enumerate}
\def\labelenumi{\arabic{enumi}.}
\setcounter{enumi}{2}
\tightlist
\item
  Hitung luas segitiga tersebut.
\end{enumerate}

Hitung luas ABC:

\[L_{\triangle ABC}= \frac{1}{2}AD.BC.\]\textgreater norm(A-D)*norm(B-C)/2 // AD=norm(A-D), BC=norm(B-C)

\begin{verbatim}
0.5
\end{verbatim}

\textgreater areaTriangle(A,B,C)

\begin{verbatim}
3
\end{verbatim}

\textgreater distance(A,D)*distance(B,C)/2

\begin{verbatim}
0.5
\end{verbatim}

\begin{enumerate}
\def\labelenumi{\arabic{enumi}.}
\setcounter{enumi}{3}
\tightlist
\item
  Tunjukkan bahwa garis bagi sudut yang ke tiga juga melalui titik pusat lingkaran dalam.
\end{enumerate}

\textgreater l=angleBisector(A,C,B);

\textgreater g=angleBisector(C,A,B);

\textgreater P=lineIntersection(l,g)

\begin{verbatim}
[1.79618,  1.744]
\end{verbatim}

\textgreater color(5); plotLine(l); plotLine(g); color(1);

\textgreater plotPoint(P,``P'');

\textgreater plotCircle(circleWithCenter(P,r),``Lingkaran dalam segitiga ABC''):

\begin{figure}
\centering
\pandocbounded{\includegraphics[keepaspectratio]{images/Haifa Azka_23030530097 (APLIKOM pekan 11-12)-007.png}}
\caption{images/Haifa\%20Azka\_23030530097\%20(APLIKOM\%20pekan\%2011-12)-007.png}
\end{figure}

\begin{enumerate}
\def\labelenumi{\arabic{enumi}.}
\setcounter{enumi}{4}
\tightlist
\item
  Gambar jari-jari lingkaran dalam.
\end{enumerate}

\textgreater O, R

\begin{verbatim}
[1.5,  1.5]
1.58113883008
\end{verbatim}

\textgreater r=norm(P-projectToLine(P,lineThrough(A,B)))

\begin{verbatim}
0.744001939852
\end{verbatim}

\begin{enumerate}
\def\labelenumi{\arabic{enumi}.}
\setcounter{enumi}{5}
\tightlist
\item
  Hitung luas lingkaran luar dan luas lingkaran dalam segitiga ABC. Adakah hubungan antara luas kedua lingkaran tersebut dengan luas segitiga ABC?
\end{enumerate}

\textgreater LU=pi*R\^{}2

\begin{verbatim}
7.85398163397
\end{verbatim}

\chapter{Contoh 2: Geometri Smbolik}\label{contoh-2-geometri-smbolik}

Kita dapat menghitung geometri eksak dan simbolik menggunakan Maxima.

File geometri.e menyediakan fungsi yang sama (dan lebih banyak lagi) di Maxima. Namun, sekarang kita dapat menggunakan perhitungan simbolik.

\textgreater A \&= {[}1,0{]}; B \&= {[}0,1{]}; C \&= {[}2,2{]}; // menentukan tiga titik A, B, C

Fungsi garis dan lingkaran berfungsi sama seperti fungsi Euler, namun menyediakan komputasi simbolik.

\textgreater c \&= lineThrough(B,C) // c=BC

\begin{verbatim}
                             [- 1, 2, 2]
\end{verbatim}

Kita bisa mendapatkan persamaan garis dengan mudah.

\textgreater\$getLineEquation(c,x,y), \$solve(\%,y) \textbar{} expand // persamaan garis c

\[2\,y-x=2\]\[\left[ y=\frac{x}{2}+1 \right] \]\textgreater\$getLineEquation(lineThrough({[}x1,y1{]},{[}x2,y2{]}),x,y), \$solve(\%,y) // persamaan garis melalui(x1, y1) dan (x2, y2)

\[x\,\left({\it y_1}-{\it y_2}\right)+\left({\it x_2}-{\it x_1}
 \right)\,y={\it x_1}\,\left({\it y_1}-{\it y_2}\right)+\left(
 {\it x_2}-{\it x_1}\right)\,{\it y_1}\]\[\left[ y=\frac{-\left({\it x_1}-x\right)\,{\it y_2}-\left(x-
 {\it x_2}\right)\,{\it y_1}}{{\it x_2}-{\it x_1}} \right] \]\textgreater\$getLineEquation(lineThrough(A,{[}x1,y1{]}),x,y) // persamaan garis melalui A dan (x1, y1)

\[\left({\it x_1}-1\right)\,y-x\,{\it y_1}=-{\it y_1}\]\textgreater h \&= perpendicular(A,lineThrough(B,C)) // h melalui A tegak lurus BC

\begin{verbatim}
                              [2, 1, 2]
\end{verbatim}

\textgreater Q \&= lineIntersection(c,h) // Q titik potong garis c=BC dan h

\begin{verbatim}
                                 2  6
                                [-, -]
                                 5  5
\end{verbatim}

\textgreater\$projectToLine(A,lineThrough(B,C)) // proyeksi A pada BC

\[\left[ \frac{2}{5} , \frac{6}{5} \right] \]\textgreater\$distance(A,Q) // jarak AQ

\[\frac{3}{\sqrt{5}}\]\textgreater cc \&= circleThrough(A,B,C); \$cc // (titik pusat dan jari-jari) lingkaran melalui A, B, C

\[\left[ \frac{7}{6} , \frac{7}{6} , \frac{5}{3\,\sqrt{2}} \right] \]\textgreater r\&=getCircleRadius(cc); \$r , \$float(r) // tampilkan nilai jari-jari

\[\frac{5}{3\,\sqrt{2}}\]\[1.178511301977579\]\textgreater\$computeAngle(A,C,B) // nilai \textless ACB

\[\arccos \left(\frac{4}{5}\right)\]\textgreater\$solve(getLineEquation(angleBisector(A,C,B),x,y),y){[}1{]} // persamaan garis bagi \textless ACB

\[y=x\]\textgreater P \&= lineIntersection(angleBisector(A,C,B),angleBisector(C,B,A)); \$P // titik potong 2 garis bagi sudut

\[\left[ \frac{\sqrt{2}\,\sqrt{5}+2}{6} , \frac{\sqrt{2}\,\sqrt{5}+2 }{6} \right]\]\textgreater P() // hasilnya sama dengan perhitungan sebelumnya

\begin{verbatim}
[0.86038,  0.86038]
\end{verbatim}

\section{Perpotongan Garis dan Lingkaran}\label{perpotongan-garis-dan-lingkaran}

Tentu saja, kita juga bisa memotong garis dengan lingkaran, dan lingkaran dengan lingkaran.

\textgreater A \&:= {[}1,0{]}; c=circleWithCenter(A,4);

\textgreater B \&:= {[}1,2{]}; C \&:= {[}2,1{]}; l=lineThrough(B,C);

\textgreater setPlotRange(5); plotCircle(c); plotLine(l);

Perpotongan garis dengan lingkaran menghasilkan dua titik dan jumlah titik perpotongan.

\textgreater\{P1,P2,f\}=lineCircleIntersections(l,c);

\textgreater P1, P2, f

\begin{verbatim}
[4.64575,  -1.64575]
[-0.645751,  3.64575]
2
\end{verbatim}

\textgreater plotPoint(P1); plotPoint(P2):

\begin{figure}
\centering
\pandocbounded{\includegraphics[keepaspectratio]{images/Haifa Azka_23030530097 (APLIKOM pekan 11-12)-021.png}}
\caption{images/Haifa\%20Azka\_23030530097\%20(APLIKOM\%20pekan\%2011-12)-021.png}
\end{figure}

Hal yang sama di Maxima.

\textgreater c \&= circleWithCenter(A,4) // lingkaran dengan pusat A jari-jari 4

\begin{verbatim}
                              [1, 0, 4]
\end{verbatim}

\textgreater l \&= lineThrough(B,C) // garis l melalui B dan C

\begin{verbatim}
                              [1, 1, 3]
\end{verbatim}

\textgreater\$lineCircleIntersections(l,c) \textbar{} radcan, // titik potong lingkaran c dan garis l

\[\left[ \left[ \sqrt{7}+2 , 1-\sqrt{7} \right]  , \left[ 2-\sqrt{7}, \sqrt{7}+1 \right]  \right]\]Akan ditunjukkan bahwa sudut-sudut yang menghadap bsuusr yang sama adalah sama besar.

\textgreater C=A+normalize({[}-2,-3{]})*4; plotPoint(C); plotSegment(P1,C); plotSegment(P2,C);

\textgreater degprint(computeAngle(P1,C,P2))

\begin{verbatim}
69°17'42.68''
\end{verbatim}

\textgreater C=A+normalize({[}-4,-3{]})*4; plotPoint(C); plotSegment(P1,C); plotSegment(P2,C);

\textgreater degprint(computeAngle(P1,C,P2))

\begin{verbatim}
69°17'42.68''
\end{verbatim}

\textgreater insimg;

\begin{figure}
\centering
\pandocbounded{\includegraphics[keepaspectratio]{images/Haifa Azka_23030530097 (APLIKOM pekan 11-12)-023.png}}
\caption{images/Haifa\%20Azka\_23030530097\%20(APLIKOM\%20pekan\%2011-12)-023.png}
\end{figure}

\section{Garis Sumbu}\label{garis-sumbu}

Berikut adalah langkah-langkah menggambar garis sumbu ruas garis AB:

\begin{enumerate}
\def\labelenumi{\arabic{enumi}.}
\item
  Gambar lingkaran dengan pusat A melalui B.
\item
  Gambar lingkaran dengan pusat B melalui A.
\item
  Tarik garis melallui kedua titik potong kedua lingkaran tersebut. Garis ini merupakan garis sumbu (melalui titik tengah dan tegak lurus) AB.
\end{enumerate}

\textgreater A={[}2,2{]}; B={[}-1,-2{]};

\textgreater c1=circleWithCenter(A,distance(A,B));

\textgreater c2=circleWithCenter(B,distance(A,B));

\textgreater\{P1,P2,f\}=circleCircleIntersections(c1,c2);

\textgreater l=lineThrough(P1,P2);

\textgreater setPlotRange(5); plotCircle(c1); plotCircle(c2);

\textgreater plotPoint(A); plotPoint(B); plotSegment(A,B); plotLine(l):

\begin{figure}
\centering
\pandocbounded{\includegraphics[keepaspectratio]{images/Haifa Azka_23030530097 (APLIKOM pekan 11-12)-024.png}}
\caption{images/Haifa\%20Azka\_23030530097\%20(APLIKOM\%20pekan\%2011-12)-024.png}
\end{figure}

Selanjutnya kita melakukan hal yang sama di Maxima dengan koordinat umum.

\textgreater A \&= {[}a1,a2{]}; B \&= {[}b1,b2{]};

\textgreater c1 \&= circleWithCenter(A,distance(A,B));

\textgreater c2 \&= circleWithCenter(B,distance(A,B));

\textgreater P \&= circleCircleIntersections(c1,c2); P1 \&= P{[}1{]}; P2 \&= P{[}2{]};

Persamaan untuk persimpangan cukup rumit. Tapi kita bisa menyederhanakannya jika kita mencari y.

\textgreater g \&= getLineEquation(lineThrough(P1,P2),x,y);

\textgreater\$solve(g,y)

\[\left[ y=\frac{-\left(2\,{\it b_1}-2\,{\it a_1}\right)\,x+{\it b_2}^2+{\it b_1}^2-{\it a_2}^2-{\it a_1}^2}{2\,{\it b_2}-2\,{\it a_2}}\right]\]Ini memang sama dengan garis tengah tegak lurus, yang dihitung dengan cara yang sangat berbeda.

\textgreater\$solve(getLineEquation(middlePerpendicular(A,B),x,y),y)

\[\left[ y=\frac{-\left(2\,{\it b_1}-2\,{\it a_1}\right)\,x+{\it b_2}^2+{\it b_1}^2-{\it a_2}^2-{\it a_1}^2}{2\,{\it b_2}-2\,{\it a_2}} \right]\]\textgreater h \&=getLineEquation(lineThrough(A,B),x,y);

\textgreater\$solve(h,y)

\[\left[ y=\frac{\left({\it b_2}-{\it a_2}\right)\,x-{\it a_1}\,{\it b_2}+{\it a_2}\,{\it b_1}}{{\it b_1}-{\it a_1}} \right]\] Perhatikan hasil kali gradien garis g dan h adalah:

\[\frac{-(b_1-a_1)}{(b_2-a_2)}\times \frac{(b_2-a_2)}{(b_1-a_1)} = -1.\]Artinya kedua garis tegak lurus.

\chapter{Contoh 3: Rumus Heron}\label{contoh-3-rumus-heron}

Rumus Heron menyatakan bahwa luas segitiga dengan panjang sisi-sisi a, b dan c adalah:

\[L = \sqrt{s(s-a)(s-b)(s-c)}\quad \text{ dengan } s=(a+b+c)/2,\]atau bisa ditulis dalam bentuk lain:

\[L = \frac{1}{4}\sqrt{(a+b+c)(b+c-a)(a+c-b)(a+b-c)}\]Untuk membuktikan hal ini kita misalkan C(0,0), B(a,0) dan A(x,y), b=AC, c=AB. Luas segitiga ABC adalah

\[L_{\triangle ABC}=\frac{1}{2}a\times y.\]Nilai y didapat dengan menyelesaikan sistem persamaan:

\[x^2+y^2=b^2, \quad (x-a)^2+y^2=c^2.\]\textgreater setPlotRange(-1,10,-1,8); plotPoint({[}0,0{]}, ``C(0,0)''); plotPoint({[}5.5,0{]}, ``B(a,0)''); \ldots{}\\
\textgreater{} plotPoint({[}7.5,6{]}, ``A(x,y)'');

\textgreater plotSegment({[}0,0{]},{[}5.5,0{]}, ``a'',25); plotSegment({[}5.5,0{]},{[}7.5,6{]},``c'',15); \ldots{}\\
\textgreater{} plotSegment({[}0,0{]},{[}7.5,6{]},``b'',25);

\textgreater plotSegment({[}7.5,6{]},{[}7.5,0{]},``t=y'',25):

\begin{figure}
\centering
\pandocbounded{\includegraphics[keepaspectratio]{images/Haifa Azka_23030530097 (APLIKOM pekan 11-12)-033.png}}
\caption{images/Haifa\%20Azka\_23030530097\%20(APLIKOM\%20pekan\%2011-12)-033.png}
\end{figure}

\textgreater\&assume(a\textgreater0); sol \&= solve({[}x\textsuperscript{2+y}2=b\textsuperscript{2,(x-a)}2+y\textsuperscript{2=c}2{]},{[}x,y{]})

\begin{verbatim}
                                  []
\end{verbatim}

Ekstrak larutan y.

\textgreater ysol \&= y with sol{[}2{]}{[}2{]}; \$'y=sqrt(factor(ysol\^{}2))

\begin{verbatim}
Maxima said:
part: invalid index of list or matrix.
 -- an error. To debug this try: debugmode(true);

Error in:
ysol &amp;= y with sol[2][2]; $'y=sqrt(factor(ysol^2)) ...
                        ^
\end{verbatim}

Kami mendapatkan rumus Heron.

\textgreater function H(a,b,c) \&= sqrt(factor((ysol*a/2)\^{}2)); \$'H(a,b,c)=H(a,b,c)

\[H\left(a , b , \left[ 1 , 0 , 4 \right] \right)=\frac{a\,\left|   {\it ysol}\right| }{2}\]\textgreater\$'Luas=H(2,5,6) // luas segitiga dengan panjang sisi-sisi 2, 5, 6

\[{\it Luas}=\left| {\it ysol}\right| \]Tentu saja, setiap segitiga siku-siku adalah kasus yang terkenal.

\textgreater H(3,4,5) //luas segitiga siku-siku dengan panjang sisi 3, 4, 5

\begin{verbatim}
Variable or function ysol not found.
Try "trace errors" to inspect local variables after errors.
H:
    useglobal; return a*abs(ysol)/2 
Error in:
H(3,4,5) //luas segitiga siku-siku dengan panjang sisi 3, 4, 5 ...
        ^
\end{verbatim}

And it is also obvious, that this is the triangle with maximal area and the two sides 3 and 4.

\textgreater aspect (1.5); plot2d(\&H(3,4,x),1,7): // Kurva luas segitiga sengan panjang sisi 3, 4, x (1\textless= x \textless=7)

\begin{verbatim}
Variable or function ysol not found.
Error in expression: 3*abs(ysol)/2
 %ploteval:
    y0=f$(x[1],args());
adaptiveevalone:
    s=%ploteval(g$,t;args());
Try "trace errors" to inspect local variables after errors.
plot2d:
    dw/n,dw/n^2,dw/n,auto;args());
\end{verbatim}

Kasus umum juga berhasil.

\textgreater\$solve(diff(H(a,b,c)\^{}2,c)=0,c)

\begin{verbatim}
Maxima said:
diff: second argument must be a variable; found [1,0,4]
 -- an error. To debug this try: debugmode(true);

Error in:
 $solve(diff(H(a,b,c)^2,c)=0,c) ...
                              ^
\end{verbatim}

Sekarang mari kita cari himpunan semua titik di mana b+c=d untuk suatu konstanta d.~Diketahui bahwa ini adalah elips.

\textgreater s1 \&= subst(d-c,b,sol{[}2{]}); \$s1

\begin{verbatim}
Maxima said:
part: invalid index of list or matrix.
 -- an error. To debug this try: debugmode(true);

Error in:
s1 &amp;= subst(d-c,b,sol[2]); $s1 ...
                         ^
\end{verbatim}

Dan buatlah fungsinya.

\textgreater function fx(a,c,d) \&= rhs(s1{[}1{]}); \$fx(a,c,d), function fy(a,c,d) \&= rhs(s1{[}2{]}); \$fy(a,c,d)

\[0\]\pandocbounded{\includegraphics[keepaspectratio]{images/Haifa Azka_23030530097 (APLIKOM pekan 11-12)-037.png}}

Sekarang kita bisa menggambar setnya. Sisi b bervariasi dari 1 sampai 4. Diketahui bahwa kita memperoleh elips.

\textgreater aspect(1); plot2d(\&fx(3,x,5),\&fy(3,x,5),xmin=1,xmax=4,square=1):

\begin{figure}
\centering
\pandocbounded{\includegraphics[keepaspectratio]{images/Haifa Azka_23030530097 (APLIKOM pekan 11-12)-038.png}}
\caption{images/Haifa\%20Azka\_23030530097\%20(APLIKOM\%20pekan\%2011-12)-038.png}
\end{figure}

Kita dapat memeriksa persamaan umum elips ini, yaitu.

\&\&\frac{(x-x_m)^2}{u^2}+\frac{(y-y_m)}{v^2}=1,

dimana (xm,ym) adalah pusat, dan u dan v adalah setengah sumbu.

\textgreater\$ratsimp((fx(a,c,d)-a/2)\textsuperscript{2/u}2+fy(a,c,d)\textsuperscript{2/v}2 with {[}u=d/2,v=sqrt(d\textsuperscript{2-a}2)/2{]})

\[\frac{a^2}{d^2}\]Kita melihat bahwa tinggi dan luas segitiga adalah maksimal untuk x=0. Jadi luas segitiga dengan a+b+c=d adalah maksimal jika segitiga tersebut sama sisi. Kami ingin memperolehnya secara analitis.

\textgreater eqns \&= {[}diff(H(a,b,d-(a+b))\textsuperscript{2,a)=0,diff(H(a,b,d-(a+b))}2,b)=0{]}; \$eqns

\[\left[ \frac{a\,{\it ysol}^2}{2}=0 , 0=0 \right]\]Kita mendapatkan nilai minimum yang dimiliki oleh segitiga dengan salah satu sisinya 0, dan solusinya a=b=c=d/3.

\textgreater\$solve(eqns,{[}a,b{]})

\[\left[ \left[ a=0 , b={\it \%r_3} \right]  \right]\]Ada juga metode Lagrange, yang memaksimalkan H(a,b,c)\^{}2 terhadap a+b+d=d.

\textgreater\&solve({[}diff(H(a,b,c)\textsuperscript{2,a)=la,diff(H(a,b,c)}2,b)=la, \ldots{}\\
\textgreater{} diff(H(a,b,c)\^{}2,c)=la,a+b+c=d{]},{[}a,b,c,la{]})

\begin{verbatim}
Maxima said:
diff: second argument must be a variable; found [1,0,4]
 -- an error. To debug this try: debugmode(true);

Error in:
... la,    diff(H(a,b,c)^2,c)=la,a+b+c=d],[a,b,c,la]) ...
                                                     ^
\end{verbatim}

Kita bisa membuat plot situasinya

Pertama atur poin di Maxima.

\textgreater A \&= at({[}x,y{]},sol{[}2{]}); \$A

\begin{verbatim}
Maxima said:
part: invalid index of list or matrix.
 -- an error. To debug this try: debugmode(true);

Error in:
A &amp;= at([x,y],sol[2]); $A ...
                     ^
\end{verbatim}

\textgreater B \&= {[}0,0{]}; \$B, C \&= {[}a,0{]}; \$C

\[\left[ a , 0 \right]\]\pandocbounded{\includegraphics[keepaspectratio]{images/Haifa Azka_23030530097 (APLIKOM pekan 11-12)-043.png}}

Kemudian atur rentang plot, dan plot titik-titiknya.

\textgreater setPlotRange(0,5,-2,3); \ldots{}\\
\textgreater{} a=4; b=3; c=2; \ldots{}\\
\textgreater{} plotPoint(mxmeval(``B''),``B''); plotPoint(mxmeval(``C''),``C''); \ldots{}\\
\textgreater{} plotPoint(mxmeval(``A''),``A''):

\begin{figure}
\centering
\pandocbounded{\includegraphics[keepaspectratio]{images/Haifa Azka_23030530097 (APLIKOM pekan 11-12)-044.png}}
\caption{images/Haifa\%20Azka\_23030530097\%20(APLIKOM\%20pekan\%2011-12)-044.png}
\end{figure}

Plot segmennya.

\textgreater plotSegment(mxmeval(``A''),mxmeval(``C'')); \ldots{}\\
\textgreater{} plotSegment(mxmeval(``B''),mxmeval(``C'')); \ldots{}\\
\textgreater{} plotSegment(mxmeval(``B''),mxmeval(``A'')):

\begin{figure}
\centering
\pandocbounded{\includegraphics[keepaspectratio]{images/Haifa Azka_23030530097 (APLIKOM pekan 11-12)-045.png}}
\caption{images/Haifa\%20Azka\_23030530097\%20(APLIKOM\%20pekan\%2011-12)-045.png}
\end{figure}

Hitung garis tengah tegak lurus di Maxima.

\textgreater h \&= middlePerpendicular(A,B); g \&= middlePerpendicular(B,C);

Dan pusat lingkarannya.

\textgreater U \&= lineIntersection(h,g);

Kita mendapatkan rumus jari-jari lingkaran luar.

\textgreater\&assume(a\textgreater0,b\textgreater0,c\textgreater0); \$distance(U,B) \textbar{} radcan

\[\frac{\sqrt{a^2+2\,a+2}}{\sqrt{2}}\]Mari kita tambahkan ini ke dalam plot.

\textgreater plotPoint(U()); \ldots{}\\
\textgreater{} plotCircle(circleWithCenter(mxmeval(``U''),mxmeval(``distance(U,C)''))):

\begin{figure}
\centering
\pandocbounded{\includegraphics[keepaspectratio]{images/Haifa Azka_23030530097 (APLIKOM pekan 11-12)-047.png}}
\caption{images/Haifa\%20Azka\_23030530097\%20(APLIKOM\%20pekan\%2011-12)-047.png}
\end{figure}

Dengan menggunakan geometri, kita memperoleh rumus sederhana

\[\frac{a}{\sin(\alpha)}=2r\]

untuk radius. Kita bisa cek, apakah ini benar adanya pada Maxima. Maxima akan memfaktorkan ini hanya jika kita mengkuadratkannya.

\textgreater\$c\textsuperscript{2/sin(computeAngle(A,B,C))}2 \textbar{} factor

\[\left[ 2 , 0 , 32 \right] \]\# Contoh 4: Garis Euler dan Parabola

Garis Euler adalah garis yang ditentukan dari sembarang segitiga yang tidak sama sisi. Merupakan garis tengah segitiga, dan melewati beberapa titik penting yang ditentukan dari segitiga, antara lain ortocenter, sirkumcenter, centroid, titik Exeter dan pusat lingkaran sembilan titik segitiga.

Untuk demonstrasinya, kita menghitung dan memplot garis Euler dalam sebuah segitiga.

Pertama, kita mendefinisikan sudut-sudut segitiga di Euler. Kami menggunakan definisi, yang terlihat dalam ekspresi simbolik.

\textgreater A::={[}-1,-1{]}; B::={[}2,0{]}; C::={[}1,2{]};

Untuk memplot objek geometris, kita menyiapkan area plot, dan menambahkan titik ke dalamnya. Semua plot objek geometris ditambahkan ke plot saat ini.

\textgreater setPlotRange(3); plotPoint(A,``A''); plotPoint(B,``B''); plotPoint(C,``C'');

Kita juga bisa menjumlahkan sisi-sisi segitiga.

\textgreater plotSegment(A,B,``\,``); plotSegment(B,C,''``); plotSegment(C,A,''\,``):

\begin{figure}
\centering
\pandocbounded{\includegraphics[keepaspectratio]{images/Haifa Azka_23030530097 (APLIKOM pekan 11-12)-049.png}}
\caption{images/Haifa\%20Azka\_23030530097\%20(APLIKOM\%20pekan\%2011-12)-049.png}
\end{figure}

Berikut luas segitiga menggunakan rumus determinan. Tentu saja kami harus mengambil nilai absolut dari hasil ini.

\textgreater\$areaTriangle(A,B,C)

\[-\frac{7}{2}\]Kita dapat menghitung koefisien sisi c.

\textgreater c \&= lineThrough(A,B)

\begin{verbatim}
                            [- 1, 3, - 2]
\end{verbatim}

Dan dapatkan juga rumus untuk baris ini.

\textgreater\$getLineEquation(c,x,y)

\[3\,y-x=-2\]Untuk bentuk Hesse, kita perlu menentukan sebuah titik, sehingga titik tersebut berada di sisi positif dari Hesseform. Memasukkan titik akan menghasilkan jarak positif ke garis.

\textgreater\$getHesseForm(c,x,y,C), \$at(\%,{[}x=C{[}1{]},y=C{[}2{]}{]})

\[\frac{7}{\sqrt{10}}\]\pandocbounded{\includegraphics[keepaspectratio]{images/Haifa Azka_23030530097 (APLIKOM pekan 11-12)-053.png}}

Sekarang kita menghitung lingkaran luar ABC.

\textgreater LL \&= circleThrough(A,B,C); \$getCircleEquation(LL,x,y)

\[\left(y-\frac{5}{14}\right)^2+\left(x-\frac{3}{14}\right)^2=\frac{  325}{98}\]\textgreater O \&= getCircleCenter(LL); \$O

\[\left[ \frac{3}{14} , \frac{5}{14} \right] \]Plot lingkaran dan pusatnya. Cu dan U bersifat simbolis. Kami mengevaluasi ekspresi ini untuk Euler.

\textgreater plotCircle(LL()); plotPoint(O(),``O''):

\begin{figure}
\centering
\pandocbounded{\includegraphics[keepaspectratio]{images/Haifa Azka_23030530097 (APLIKOM pekan 11-12)-056.png}}
\caption{images/Haifa\%20Azka\_23030530097\%20(APLIKOM\%20pekan\%2011-12)-056.png}
\end{figure}

Kita dapat menghitung perpotongan ketinggian di ABC (ortocenter) secara numerik dengan perintah berikut.

\textgreater H \&= lineIntersection(perpendicular(A,lineThrough(C,B)),\ldots{}\\
\textgreater{} perpendicular(B,lineThrough(A,C))); \$H

\[\left[ \frac{11}{7} , \frac{2}{7} \right] \]Sekarang kita dapat menghitung garis segitiga Euler.

\textgreater el \&= lineThrough(H,O); \$getLineEquation(el,x,y)

\[-\frac{19\,y}{14}-\frac{x}{14}=-\frac{1}{2}\]Tambahkan ke plot kami.

\textgreater plotPoint(H(),``H''); plotLine(el(),``Garis Euler''):

\begin{figure}
\centering
\pandocbounded{\includegraphics[keepaspectratio]{images/Haifa Azka_23030530097 (APLIKOM pekan 11-12)-059.png}}
\caption{images/Haifa\%20Azka\_23030530097\%20(APLIKOM\%20pekan\%2011-12)-059.png}
\end{figure}

Pusat gravitasi seharusnya berada di garis ini.

\textgreater M \&= (A+B+C)/3; \$getLineEquation(el,x,y) with

\[-\frac{1}{2}=-\frac{1}{2}\]\textgreater plotPoint(M(),``M''): // titik berat

\begin{figure}
\centering
\pandocbounded{\includegraphics[keepaspectratio]{images/Haifa Azka_23030530097 (APLIKOM pekan 11-12)-061.png}}
\caption{images/Haifa\%20Azka\_23030530097\%20(APLIKOM\%20pekan\%2011-12)-061.png}
\end{figure}

Teorinya memberitahu kita MH=2*MO. Kita perlu menyederhanakan dengan radcan untuk mencapai hal ini.

\textgreater\$distance(M,H)/distance(M,O)\textbar radcan

\[2\]Fungsinya mencakup fungsi untuk sudut juga.

\textgreater\$computeAngle(A,C,B), degprint(\%())

\[\arccos \left(\frac{4}{\sqrt{5}\,\sqrt{13}}\right)\] 60°15'18.43'\,'

Persamaan pusat lingkaran tidak terlalu bagus.

\textgreater Q \&= lineIntersection(angleBisector(A,C,B),angleBisector(C,B,A))\textbar radcan; \$Q

\[\left[ \frac{\left(2^{\frac{3}{2}}+1\right)\,\sqrt{5}\,\sqrt{13}-15  \,\sqrt{2}+3}{14} , \frac{\left(\sqrt{2}-3\right)\,\sqrt{5}\,\sqrt{  13}+5\,2^{\frac{3}{2}}+5}{14} \right] \]Mari kita hitung juga ekspresi jari-jari lingkaran yang tertulis.

\textgreater r \&= distance(Q,projectToLine(Q,lineThrough(A,B)))\textbar ratsimp; \$r

\[\frac{\sqrt{\left(-41\,\sqrt{2}-31\right)\,\sqrt{5}\,\sqrt{13}+115  \,\sqrt{2}+614}}{7\,\sqrt{2}}\]\textgreater LD \&= circleWithCenter(Q,r); // Lingkaran dalam

Mari kita tambahkan ini ke dalam plot.

\textgreater color(5); plotCircle(LD()):

\begin{figure}
\centering
\pandocbounded{\includegraphics[keepaspectratio]{images/Haifa Azka_23030530097 (APLIKOM pekan 11-12)-066.png}}
\caption{images/Haifa\%20Azka\_23030530097\%20(APLIKOM\%20pekan\%2011-12)-066.png}
\end{figure}

\section{Parabola}\label{parabola}

Selanjutnya akan dicari persamaan tempat kedudukan titik-titik yang berjarak sama ke titik C dan ke garis AB.

\textgreater p \&= getHesseForm(lineThrough(A,B),x,y,C)-distance({[}x,y{]},C); \$p='0

\[\frac{3\,y-x+2}{\sqrt{10}}-\sqrt{\left(2-y\right)^2+\left(1-x  \right)^2}=0\]Persamaan tersebut dapat digambar menjadi satu dengan gambar sebelumnya.

\textgreater plot2d(p,level=0,add=1,contourcolor=6):

\begin{figure}
\centering
\pandocbounded{\includegraphics[keepaspectratio]{images/Haifa Azka_23030530097 (APLIKOM pekan 11-12)-068.png}}
\caption{images/Haifa\%20Azka\_23030530097\%20(APLIKOM\%20pekan\%2011-12)-068.png}
\end{figure}

Ini seharusnya merupakan suatu fungsi, tetapi pemecah default Maxima hanya dapat menemukan solusinya, jika kita mengkuadratkan persamaannya. Akibatnya, kami mendapatkan solusi palsu.

\textgreater akar \&= solve(getHesseForm(lineThrough(A,B),x,y,C)\textsuperscript{2-distance({[}x,y{]},C)}2,y)

\begin{verbatim}
        [y = - 3 x - sqrt(70) sqrt(9 - 2 x) + 26, 
                              y = - 3 x + sqrt(70) sqrt(9 - 2 x) + 26]
\end{verbatim}

Solusi pertama adalah

maksimal: akar{[}1{]}

Menambahkan solusi pertama pada plot menunjukkan, bahwa itu memang jalan yang kita cari. Teorinya memberitahu kita bahwa itu adalah parabola yang diputar.

\textgreater plot2d(\&rhs(akar{[}1{]}),add=1):

\begin{figure}
\centering
\pandocbounded{\includegraphics[keepaspectratio]{images/Haifa Azka_23030530097 (APLIKOM pekan 11-12)-069.png}}
\caption{images/Haifa\%20Azka\_23030530097\%20(APLIKOM\%20pekan\%2011-12)-069.png}
\end{figure}

\textgreater function g(x) \&= rhs(akar{[}1{]}); \$'g(x)= g(x)// fungsi yang mendefinisikan kurva di atas

\[g\left(x\right)=-3\,x-\sqrt{70}\,\sqrt{9-2\,x}+26\]\textgreater T \&={[}-1, g(-1){]}; // ambil sebarang titik pada kurva tersebut

\textgreater dTC \&= distance(T,C); \$fullratsimp(dTC), \$float(\%) // jarak T ke C

\[2.135605779339061\]\pandocbounded{\includegraphics[keepaspectratio]{images/Haifa Azka_23030530097 (APLIKOM pekan 11-12)-072.png}}

\textgreater U \&= projectToLine(T,lineThrough(A,B)); \$U // proyeksi T pada garis AB

\[\left[ \frac{80-3\,\sqrt{11}\,\sqrt{70}}{10} , \frac{20-\sqrt{11}\,  \sqrt{70}}{10} \right] \]\textgreater dU2AB \&= distance(T,U); \$fullratsimp(dU2AB), \$float(\%) // jatak T ke AB

\[2.135605779339061\]\pandocbounded{\includegraphics[keepaspectratio]{images/Haifa Azka_23030530097 (APLIKOM pekan 11-12)-075.png}}

Ternyata jarak T ke C sama dengan jarak T ke AB. Coba Anda pilih titik T yang lain dan ulangi perhitungan-perhitungan di atas untuk menunjukkan bahwa hasilnya juga sama.

\chapter{Contoh 5: Trigonometri Rasional}\label{contoh-5-trigonometri-rasional}

Hal ini terinspirasi dari ceramah N.J.Wildberger. Dalam bukunya ``Divine Proportions'', Wildberger mengusulkan untuk mengganti gagasan klasik tentang jarak dan sudut dengan kuadran dan penyebaran. Dengan menggunakan hal ini, memang mungkin untuk menghindari fungsi trigonometri dalam banyak contoh, dan tetap ``rasional''.

Berikut ini, saya memperkenalkan konsep, dan memecahkan beberapa masalah. Saya menggunakan perhitungan simbolik Maxima di sini, yang menyembunyikan keunggulan utama trigonometri rasional yaitu perhitungan hanya dapat dilakukan dengan kertas dan pensil. Anda diundang untuk memeriksa hasilnya tanpa komputer.

Intinya adalah perhitungan rasional simbolik seringkali memberikan hasil yang sederhana. Sebaliknya, trigonometri klasik menghasilkan hasil trigonometri yang rumit, yang hanya mengevaluasi perkiraan numerik saja.

\textgreater load geometry;

Untuk pengenalan pertama, kami menggunakan segitiga siku-siku dengan proporsi Mesir yang terkenal 3, 4 dan 5. Perintah berikut adalah perintah Euler untuk memplot geometri bidang yang terdapat dalam file Euler ``geometry.e''.

\textgreater C\&:={[}0,0{]}; A\&:={[}4,0{]}; B\&:={[}0,3{]}; \ldots{}\\
\textgreater{} setPlotRange(-1,5,-1,5); \ldots{}\\
\textgreater{} plotPoint(A,``A''); plotPoint(B,``B''); plotPoint(C,``C''); \ldots{}\\
\textgreater{} plotSegment(B,A,``c''); plotSegment(A,C,``b''); plotSegment(C,B,``a''); \ldots{}\\
\textgreater{} insimg(30);

\begin{figure}
\centering
\pandocbounded{\includegraphics[keepaspectratio]{images/Haifa Azka_23030530097 (APLIKOM pekan 11-12)-076.png}}
\caption{images/Haifa\%20Azka\_23030530097\%20(APLIKOM\%20pekan\%2011-12)-076.png}
\end{figure}

Tentu saja,

lateks: \sin(w\_a)=\frac{a}{c},

dimana wa adalah sudut di A. Cara umum untuk menghitung sudut ini adalah dengan mengambil invers dari fungsi sinus. Hasilnya adalah sudut yang tidak dapat dicerna, yang hanya dapat dicetak secara kasar.

\textgreater wa := arcsin(3/5); degprint(wa)

\begin{verbatim}
36°52'11.63''
\end{verbatim}

Trigonometri rasional mencoba menghindari hal ini.

Gagasan pertama tentang trigonometri rasional adalah kuadran, yang menggantikan jarak. Faktanya, itu hanyalah jarak yang dikuadratkan. Di bawah ini, a, b, dan c menyatakan kuadran sisi-sisinya.

Teorema Pythogoras menjadi a+b=c.

\textgreater a \&= 3\^{}2; b \&= 4\^{}2; c \&= 5\^{}2; \&a+b=c

\begin{verbatim}
                               25 = 25
\end{verbatim}

Pengertian trigonometri rasional yang kedua adalah penyebaran. Penyebaran mengukur pembukaan antar garis. Nilainya 0 jika garisnya sejajar, dan 1 jika garisnya persegi panjang. Ini adalah kuadrat sinus sudut antara dua garis.

Luas garis AB dan AC pada gambar di atas didefinisikan sebagai

lateks: s\_a = \sin(\alpha)\^{}2 = \frac{a}{c},

dimana a dan c adalah kuadran suatu segitiga siku-siku yang salah satu sudutnya berada di A.

\textgreater sa \&= a/c; \$sa

\[\frac{9}{25}\]Tentu saja ini lebih mudah dihitung daripada sudutnya. Namun Anda kehilangan properti bahwa sudut dapat ditambahkan dengan mudah.

Tentu saja, kita dapat mengonversi nilai perkiraan sudut wa menjadi sprad, dan mencetaknya sebagai pecahan.

\textgreater fracprint(sin(wa)\^{}2)

\begin{verbatim}
9/25
\end{verbatim}

Hukum kosinus trgonometri klasik diterjemahkan menjadi ``hukum silang'' berikut.

lateks:(c+b-a)\^{}2 = 4 b c , (1-s\_a)

Di sini a, b, dan c adalah kuadran sisi-sisi segitiga, dan sa adalah jarak di sudut A. Sisi a, seperti biasa, berhadapan dengan sudut A.

Hukum-hukum ini diterapkan dalam file geometri.e yang kami muat ke Euler.

\textgreater\$crosslaw(aa,bb,cc,saa)

\[\left[ \left({\it bb}-{\it aa}+\frac{7}{6}\right)^2 , \left(  {\it bb}-{\it aa}+\frac{7}{6}\right)^2 , \left({\it bb}-{\it aa}+  \frac{5}{3\,\sqrt{2}}\right)^2 \right] =\left[ \frac{14\,{\it bb}\,  \left(1-{\it saa}\right)}{3} , \frac{14\,{\it bb}\,\left(1-{\it saa}  \right)}{3} , \frac{5\,2^{\frac{3}{2}}\,{\it bb}\,\left(1-{\it saa}  \right)}{3} \right] \]Dalam kasus kami, kami mendapatkan

\textgreater\$crosslaw(a,b,c,sa)

\[1024=1024\]Mari kita gunakan hukum silang ini untuk mencari penyebaran di A. Untuk melakukannya, kita buat hukum silang untuk kuadran a, b, dan c, dan selesaikan untuk penyebaran yang tidak diketahui sa.

Anda bisa melakukannya dengan tangan dengan mudah, tapi saya menggunakan Maxima. Tentu saja, kami mendapatkan hasilnya, kami sudah mendapatkannya.

\textgreater\$crosslaw(a,b,c,x), \$solve(\%,x)

\[\left[ x=\frac{9}{25} \right] \]\pandocbounded{\includegraphics[keepaspectratio]{images/Haifa Azka_23030530097 (APLIKOM pekan 11-12)-081.png}}

Kami sudah mengetahui hal ini. Pengertian penyebaran merupakan kasus khusus dari hukum silang.

Kita juga dapat menyelesaikannya untuk persamaan umum a,b,c.~Hasilnya adalah rumus yang menghitung penyebaran sudut suatu segitiga dengan mengetahui kuadran ketiga sisinya.

\textgreater\$solve(crosslaw(aa,bb,cc,x),x)

\[\left[ \left[ \frac{168\,{\it bb}\,x+36\,{\it bb}^2+\left(-72\,  {\it aa}-84\right)\,{\it bb}+36\,{\it aa}^2-84\,{\it aa}+49}{36} ,   \frac{168\,{\it bb}\,x+36\,{\it bb}^2+\left(-72\,{\it aa}-84\right)  \,{\it bb}+36\,{\it aa}^2-84\,{\it aa}+49}{36} , \frac{15\,2^{\frac{  5}{2}}\,{\it bb}\,x+18\,{\it bb}^2+\left(-36\,{\it aa}-15\,2^{\frac{  3}{2}}\right)\,{\it bb}+18\,{\it aa}^2-15\,2^{\frac{3}{2}}\,{\it aa}  +25}{18} \right] =0 \right] \]Kita bisa membuat fungsi dari hasilnya. Fungsi seperti itu sudah didefinisikan dalam file geometri.e Euler.

\textgreater\$spread(a,b,c)

\[\frac{9}{25}\]Sebagai contoh, kita dapat menggunakannya untuk menghitung sudut segitiga dengan sisi-sisinya

lateks: a, \quad a, \quad \frac{4a}{7}

Hasilnya rasional, yang tidak mudah didapat jika kita menggunakan trigonometri klasik.

\textgreater\$spread(a,a,4*a/7)

\[\frac{6}{7}\]Ini adalah sudut dalam derajat.

\textgreater degprint(arcsin(sqrt(6/7)))

\begin{verbatim}
67°47'32.44''
\end{verbatim}

\section{Contoh Lain}\label{contoh-lain}

Sekarang, mari kita coba contoh lebih lanjut.

Kita tentukan tiga sudut segitiga sebagai berikut.

\textgreater A\&:={[}1,2{]}; B\&:={[}4,3{]}; C\&:={[}0,4{]}; \ldots{}\\
\textgreater{} setPlotRange(-1,5,1,7); \ldots{}\\
\textgreater{} plotPoint(A,``A''); plotPoint(B,``B''); plotPoint(C,``C''); \ldots{}\\
\textgreater{} plotSegment(B,A,``c''); plotSegment(A,C,``b''); plotSegment(C,B,``a''); \ldots{}\\
\textgreater{} insimg;

\begin{figure}
\centering
\pandocbounded{\includegraphics[keepaspectratio]{images/Haifa Azka_23030530097 (APLIKOM pekan 11-12)-085.png}}
\caption{images/Haifa\%20Azka\_23030530097\%20(APLIKOM\%20pekan\%2011-12)-085.png}
\end{figure}

Dengan menggunakan Pythogoras, mudah untuk menghitung jarak antara dua titik. Saya pertama kali menggunakan fungsi jarak file Euler untuk geometri. Fungsi jarak menggunakan geometri klasik.

\textgreater\$distance(A,B)

\[\sqrt{10}\]Euler juga memuat fungsi kuadran antara dua titik.

Pada contoh berikut, karena c+b bukan a, maka segitiga tersebut bukan persegi panjang.

\textgreater c \&= quad(A,B); \$c, b \&= quad(A,C); \$b, a \&= quad(B,C); \$a,

\[17\]\pandocbounded{\includegraphics[keepaspectratio]{images/Haifa Azka_23030530097 (APLIKOM pekan 11-12)-088.png}}

\begin{figure}
\centering
\pandocbounded{\includegraphics[keepaspectratio]{images/Haifa Azka_23030530097 (APLIKOM pekan 11-12)-089.png}}
\caption{images/Haifa\%20Azka\_23030530097\%20(APLIKOM\%20pekan\%2011-12)-089.png}
\end{figure}

Pertama, mari kita hitung sudut tradisional. Fungsi computeAngle menggunakan metode biasa berdasarkan perkalian titik dua vektor. Hasilnya adalah beberapa perkiraan floating point.

\[A=<1,2>\quad B=<4,3>,\quad C=<0,4>\]\[\mathbf{a}=C-B=<-4,1>,\quad \mathbf{c}=A-B=<-3,-1>,\quad \beta=\angle ABC\]\[\mathbf{a}.\mathbf{c}=|\mathbf{a}|.|\mathbf{c}|\cos \beta\]\[\cos \angle ABC =\cos\beta=\frac{\mathbf{a}.\mathbf{c}}{|\mathbf{a}|.|\mathbf{c}|}=\frac{12-1}{\sqrt{17}\sqrt{10}}=\frac{11}{\sqrt{17}\sqrt{10}}\]\textgreater wb \&= computeAngle(A,B,C); \$wb, \$(wb/pi*180)()

\[\arccos \left(\frac{11}{\sqrt{10}\,\sqrt{17}}\right)\] 32.4711922908

Dengan menggunakan pensil dan kertas, kita dapat melakukan hal yang sama dengan hukum silang. Kita masukkan kuadran a, b, dan c ke dalam hukum silang dan selesaikan x.

\textgreater\$crosslaw(a,b,c,x), \$solve(\%,x), //(b+c-a)\^{}=4b.c(1-x)

\[\left[ x=\frac{49}{50} \right]\]\pandocbounded{\includegraphics[keepaspectratio]{images/Haifa Azka_23030530097 (APLIKOM pekan 11-12)-096.png}}

Yaitu, fungsi penyebaran yang didefinisikan dalam ``geometri.e''.

\textgreater sb \&= spread(b,a,c); \$sb

\[\frac{49}{170}\]Maxima mendapatkan hasil yang sama dengan menggunakan trigonometri biasa, jika kita memaksakannya. Itu menyelesaikan suku sin(arccos(\ldots)) menjadi hasil pecahan. Kebanyakan siswa tidak dapat melakukan hal ini.

\textgreater\$sin(computeAngle(A,B,C))\^{}2

\[\frac{49}{170}\]Setelah kita mendapatkan sebaran di B, kita dapat menghitung tinggi ha pada sisi a. Ingat itu

\[s_b=\frac{h_a}{c}\]berdasarkan definisi.

\textgreater ha \&= c*sb; \$ha

\[\frac{49}{17}\]Gambar berikut dihasilkan dengan program geometri C.a.R., yang dapat menggambar kuadran dan sebaran.

image: (20) Rational\_Geometry\_CaR.png

Menurut definisi, panjang ha adalah akar kuadrat dari kuadrannya.

\textgreater\$sqrt(ha)

\[\frac{7}{\sqrt{17}}\]Sekarang kita dapat menghitung luas segitiga tersebut. Jangan lupa, bahwa kita sedang berhadapan dengan kuadran!

\textgreater\$sqrt(ha)*sqrt(a)/2

\[\frac{7}{2}\]Rumus determinan biasa memberikan hasil yang sama.

\textgreater\$areaTriangle(B,A,C)

\[\frac{7}{2}\]

\section{Rumus Bangau}\label{rumus-bangau}

Sekarang, mari kita selesaikan masalah ini secara umum!

\textgreater\&remvalue(a,b,c,sb,ha);

Pertama-tama kita menghitung penyebaran di B untuk sebuah segitiga dengan sisi a, b, dan c.~Kemudian kita menghitung luas kuadrat (``quadrea''?), memfaktorkannya dengan Maxima, dan kita mendapatkan rumus Heron yang terkenal.

Memang benar, hal ini sulit dilakukan dengan pensil dan kertas.

\textgreater\$spread(b\textsuperscript{2,c}2,a\^{}2), \$factor(\%*c\textsuperscript{2*a}2/4)

\[\frac{\left(-c+b+a\right)\,\left(c-b+a\right)\,\left(c+b-a\right)\,  \left(c+b+a\right)}{16}\]\pandocbounded{\includegraphics[keepaspectratio]{images/Haifa Azka_23030530097 (APLIKOM pekan 11-12)-105.png}}

\section{Aturan Penyebaran Tiga Kali Lipat}\label{aturan-penyebaran-tiga-kali-lipat}

Kerugian dari spread adalah bahwa mereka tidak lagi hanya menambahkan sudut yang sama.

Namun, tiga spread segitiga memenuhi aturan ``triple spread'' berikut.

\textgreater\&remvalue(sa,sb,sc); \$triplespread(sa,sb,sc)

\[\left({\it sc}+{\it sb}+{\it sa}\right)^2=2\,\left({\it sc}^2+  {\it sb}^2+{\it sa}^2\right)+4\,{\it sa}\,{\it sb}\,{\it sc}\]Aturan ini berlaku untuk tiga sudut mana pun yang besarnya 180°.

\[\alpha+\beta+\gamma=\pi\]Sejak menyebarnya

\[\alpha, \pi-\alpha\]sama, aturan penyebaran tiga kali lipat juga benar, jika

\[\alpha+\beta=\gamma\]Karena penyebaran sudut negatifnya sama, maka aturan penyebaran tiga kali lipat juga berlaku, jika

\[\alpha+\beta+\gamma=0\]atau contohnya, kita dapat menghitung penyebaran sudut 60°. Ini 3/4. Namun persamaan tersebut memiliki solusi kedua, dimana semua spread adalah 0.

\textgreater\$solve(triplespread(x,x,x),x)

\[\left[ x=\frac{3}{4} , x=0 \right]\]Penyebaran 90° jelas sama dengan 1. Jika dua sudut dijumlahkan menjadi 90°, penyebarannya menyelesaikan persamaan penyebaran rangkap tiga dengan a,b,1. Dengan perhitungan berikut kita mendapatkan a+b=1.

\textgreater\$triplespread(x,y,1), \$solve(\%,x)

\[\left[ x=1-y \right]\]\pandocbounded{\includegraphics[keepaspectratio]{images/Haifa Azka_23030530097 (APLIKOM pekan 11-12)-113.png}}

Karena penyebaran 180°-t sama dengan penyebaran t, rumus penyebaran tiga kali lipat juga berlaku, jika salah satu sudut adalah jumlah atau selisih dua sudut lainnya.

Sehingga kita dapat mencari penyebaran sudut dua kali lipat tersebut. Perhatikan bahwa ada dua solusi lagi. Kami menjadikan ini sebuah fungsi.

\textgreater\$solve(triplespread(a,a,x),x), function doublespread(a) \&= factor(rhs(\%{[}1{]}))

\[\left[ x=4\,a-4\,a^2 , x=0 \right]\]\\
- 4 (a - 1) a

\section{Pembagi Sudut}\label{pembagi-sudut}

Inilah situasinya, kita sudah tahu.

\textgreater C\&:={[}0,0{]}; A\&:={[}4,0{]}; B\&:={[}0,3{]}; \ldots{}\\
\textgreater{} setPlotRange(-1,5,-1,5); \ldots{}\\
\textgreater{} plotPoint(A,``A''); plotPoint(B,``B''); plotPoint(C,``C''); \ldots{}\\
\textgreater{} plotSegment(B,A,``c''); plotSegment(A,C,``b''); plotSegment(C,B,``a''); \ldots{}\\
\textgreater{} insimg;

\begin{figure}
\centering
\pandocbounded{\includegraphics[keepaspectratio]{images/Haifa Azka_23030530097 (APLIKOM pekan 11-12)-115.png}}
\caption{images/Haifa\%20Azka\_23030530097\%20(APLIKOM\%20pekan\%2011-12)-115.png}
\end{figure}

Mari kita hitung panjang garis bagi sudut di A. Namun kita ingin menyelesaikannya secara umum a,b,c.

\textgreater\&remvalue(a,b,c);

Jadi pertama-tama kita menghitung penyebaran sudut yang dibagi dua di A, menggunakan rumus penyebaran tiga kali lipat.

Masalah dengan rumus ini muncul lagi. Ini memiliki dua solusi. Kita harus memilih yang benar. Solusi lainnya mengacu pada sudut membagi dua 180°-wa.

\textgreater\$triplespread(x,x,a/(a+b)), \$solve(\%,x), sa2 \&= rhs(\%{[}1{]}); \$sa2

\[\frac{-\sqrt{b}\,\sqrt{b+a}+b+a}{2\,b+2\,a}\]\pandocbounded{\includegraphics[keepaspectratio]{images/Haifa Azka_23030530097 (APLIKOM pekan 11-12)-117.png}}

\begin{figure}
\centering
\pandocbounded{\includegraphics[keepaspectratio]{images/Haifa Azka_23030530097 (APLIKOM pekan 11-12)-118.png}}
\caption{images/Haifa\%20Azka\_23030530097\%20(APLIKOM\%20pekan\%2011-12)-118.png}
\end{figure}

Mari kita periksa persegi panjang Mesir.

\textgreater\$sa2 with {[}a=3\textsuperscript{2,b=4}2{]}

\[\frac{1}{10}\]Kita dapat mencetak sudut dalam Euler, setelah mentransfer penyebarannya ke radian.

\textgreater wa2 := arcsin(sqrt(1/10)); degprint(wa2)

\begin{verbatim}
18°26'5.82''
\end{verbatim}

Titik P merupakan perpotongan garis bagi sudut dengan sumbu y.

\textgreater P := {[}0,tan(wa2)*4{]}

\begin{verbatim}
[0,  1.33333]
\end{verbatim}

\textgreater plotPoint(P,``P''); plotSegment(A,P):

\begin{figure}
\centering
\pandocbounded{\includegraphics[keepaspectratio]{images/Haifa Azka_23030530097 (APLIKOM pekan 11-12)-120.png}}
\caption{images/Haifa\%20Azka\_23030530097\%20(APLIKOM\%20pekan\%2011-12)-120.png}
\end{figure}

Mari kita periksa sudut dalam contoh spesifik kita.

\textgreater computeAngle(C,A,P), computeAngle(P,A,B)

\begin{verbatim}
0.321750554397
0.321750554397
\end{verbatim}

Sekarang kita menghitung panjang garis bagi AP.

Kita menggunakan teorema sinus pada segitiga APC. Teorema ini menyatakan bahwa

\[\frac{BC}{\sin(w_a)} = \frac{AC}{\sin(w_b)} = \frac{AB}{\sin(w_c)}\]berlaku di segitiga mana pun. Jika digabungkan, maka hal ini akan diterjemahkan ke dalam apa yang disebut dengan ``hukum penyebaran''

\[\frac{a}{s_a} = \frac{b}{s_b} = \frac{c}{s_b}\]dimana a,b,c menunjukkan qudrance.

Karena spread CPA adalah 1-sa2, kita memperolehnya bisa/1=b/(1-sa2) dan dapat menghitung bisa (kuadran dari garis bagi sudut).

\textgreater\&factor(ratsimp(b/(1-sa2))); bisa \&= \%; \$bisa

\[\frac{2\,b\,\left(b+a\right)}{\sqrt{b}\,\sqrt{b+a}+b+a}\]Mari kita periksa rumus ini untuk nilai-nilai Mesir kita.

\textgreater sqrt(mxmeval(``at(bisa,{[}a=3\textsuperscript{2,b=4}2{]})'')), distance(A,P)

\begin{verbatim}
4.21637021356
4.21637021356
\end{verbatim}

Kita juga bisa menghitung P menggunakan rumus spread.

\textgreater py\&=factor(ratsimp(sa2*bisa)); \$py

\[-\frac{b\,\left(\sqrt{b}\,\sqrt{b+a}-b-a\right)}{\sqrt{b}\,\sqrt{b+  a}+b+a}\]Nilainya sama dengan yang kita peroleh dengan rumus trigonometri.

\textgreater sqrt(mxmeval(``at(py,{[}a=3\textsuperscript{2,b=4}2{]})''))

\begin{verbatim}
1.33333333333
\end{verbatim}

\section{Sudut Akord}\label{sudut-akord}

Lihatlah situasi berikut.

\textgreater setPlotRange(1.2); \ldots{}\\
\textgreater{} color(1); plotCircle(circleWithCenter({[}0,0{]},1)); \ldots{}\\
\textgreater{} A:={[}cos(1),sin(1){]}; B:={[}cos(2),sin(2){]}; C:={[}cos(6),sin(6){]}; \ldots{}\\
\textgreater{} plotPoint(A,``A''); plotPoint(B,``B''); plotPoint(C,``C''); \ldots{}\\
\textgreater{} color(3); plotSegment(A,B,``c''); plotSegment(A,C,``b''); plotSegment(C,B,``a''); \ldots{}\\
\textgreater{} color(1); O:={[}0,0{]}; plotPoint(O,``0''); \ldots{}\\
\textgreater{} plotSegment(A,O); plotSegment(B,O); plotSegment(C,O,``r''); \ldots{}\\
\textgreater{} insimg;

\begin{figure}
\centering
\pandocbounded{\includegraphics[keepaspectratio]{images/Haifa Azka_23030530097 (APLIKOM pekan 11-12)-125.png}}
\caption{images/Haifa\%20Azka\_23030530097\%20(APLIKOM\%20pekan\%2011-12)-125.png}
\end{figure}

Kita dapat menggunakan Maxima untuk menyelesaikan rumus penyebaran rangkap tiga untuk sudut di pusat O untuk r. Jadi kita mendapatkan rumus jari-jari kuadrat dari perilingkaran dalam kuadran sisi-sisinya.

Kali ini, Maxima menghasilkan beberapa angka nol kompleks, yang kita abaikan.

\textgreater\&remvalue(a,b,c,r); // hapus nilai-nilai sebelumnya untuk perhitungan baru

\textgreater rabc \&= rhs(solve(triplespread(spread(b,r,r),spread(a,r,r),spread(c,r,r)),r){[}4{]}); \$rabc

\[-\frac{a\,b\,c}{c^2-2\,b\,c+a\,\left(-2\,c-2\,b\right)+b^2+a^2}\]Kita dapat menjadikannya fungsi Euler.

\textgreater function periradius(a,b,c) \&= rabc;

Mari kita periksa hasil untuk poin kita A,B,C.

\textgreater a:=quadrance(B,C); b:=quadrance(A,C); c:=quadrance(A,B);

Jari-jarinya memang 1.

\textgreater periradius(a,b,c)

\begin{verbatim}
1
\end{verbatim}

Faktanya, penyebaran CBA hanya bergantung pada b dan c.~Ini adalah teorema sudut tali busur.

\textgreater\$spread(b,a,c)*rabc \textbar{} ratsimp

\[\frac{b}{4}\]Faktanya, penyebarannya adalah b/(4r), dan kita melihat bahwa sudut tali busur b adalah setengah sudut pusatnya.

\textgreater\$doublespread(b/(4*r))-spread(b,r,r) \textbar{} ratsimp

\$\(0\)

\chapter{Contoh 6: Jarak Minimal pada Bidang}\label{contoh-6-jarak-minimal-pada-bidang}

\section{Catatan pendahuluan}\label{catatan-pendahuluan}

Fungsi yang, ke titik M pada bidang, menetapkan jarak AM antara titik tetap A dan M, mempunyai garis datar yang cukup sederhana: lingkaran berpusat di A.

\textgreater\&remvalue();

\textgreater A={[}-1,-1{]};

\textgreater function d1(x,y):=sqrt((x-A{[}1{]})\textsuperscript{2+(y-A{[}2{]})}2)

\textgreater fcontour(``d1'',xmin=-2,xmax=0,ymin=-2,ymax=0,hue=1, \ldots{}\\
\textgreater{} title=``If you see ellipses, please set your window square''):

\begin{figure}
\centering
\pandocbounded{\includegraphics[keepaspectratio]{images/Haifa Azka_23030530097 (APLIKOM pekan 11-12)-129.png}}
\caption{images/Haifa\%20Azka\_23030530097\%20(APLIKOM\%20pekan\%2011-12)-129.png}
\end{figure}

dan grafiknya juga cukup sederhana: bagian atas kerucut:

\textgreater plot3d(``d1'',xmin=-2,xmax=0,ymin=-2,ymax=0):

\begin{figure}
\centering
\pandocbounded{\includegraphics[keepaspectratio]{images/Haifa Azka_23030530097 (APLIKOM pekan 11-12)-130.png}}
\caption{images/Haifa\%20Azka\_23030530097\%20(APLIKOM\%20pekan\%2011-12)-130.png}
\end{figure}

Tentu saja minimum 0 dicapai di A.

\section{Dua poin}\label{dua-poin}

Sekarang kita lihat fungsi MA+MB dimana A dan B adalah dua titik (tetap). Merupakan ``fakta yang diketahui'' bahwa kurva tingkat berbentuk elips, titik fokusnya adalah A dan B; kecuali AB minimum yang konstan pada ruas {[}AB{]}:

\textgreater B={[}1,-1{]};

\textgreater function d2(x,y):=d1(x,y)+sqrt((x-B{[}1{]})\textsuperscript{2+(y-B{[}2{]})}2)

\textgreater fcontour(``d2'',xmin=-2,xmax=2,ymin=-3,ymax=1,hue=1):

\begin{figure}
\centering
\pandocbounded{\includegraphics[keepaspectratio]{images/Haifa Azka_23030530097 (APLIKOM pekan 11-12)-131.png}}
\caption{images/Haifa\%20Azka\_23030530097\%20(APLIKOM\%20pekan\%2011-12)-131.png}
\end{figure}

Grafiknya lebih menarik:

\textgreater plot3d(``d2'',xmin=-2,xmax=2,ymin=-3,ymax=1):

\begin{figure}
\centering
\pandocbounded{\includegraphics[keepaspectratio]{images/Haifa Azka_23030530097 (APLIKOM pekan 11-12)-132.png}}
\caption{images/Haifa\%20Azka\_23030530097\%20(APLIKOM\%20pekan\%2011-12)-132.png}
\end{figure}

Pembatasan pada garis (AB) lebih terkenal:

\textgreater plot2d(``abs(x+1)+abs(x-1)'',xmin=-3,xmax=3):

\begin{figure}
\centering
\pandocbounded{\includegraphics[keepaspectratio]{images/Haifa Azka_23030530097 (APLIKOM pekan 11-12)-133.png}}
\caption{images/Haifa\%20Azka\_23030530097\%20(APLIKOM\%20pekan\%2011-12)-133.png}
\end{figure}

\section{Tiga poin}\label{tiga-poin}

Kini segalanya menjadi lebih sederhana: Tidak diketahui secara luas bahwa MA+MB+MC mencapai nilai minimumnya pada satu titik pada bidang tersebut, namun untuk menentukannya tidaklah mudah:

\begin{enumerate}
\def\labelenumi{\arabic{enumi})}
\tightlist
\item
  Jika salah satu sudut segitiga ABC lebih dari 120° (katakanlah di A), maka sudut minimum dicapai pada titik tersebut (katakanlah AB+AC).
\end{enumerate}

Contoh:

\textgreater C={[}-4,1{]};

\textgreater function d3(x,y):=d2(x,y)+sqrt((x-C{[}1{]})\textsuperscript{2+(y-C{[}2{]})}2)

\textgreater plot3d(``d3'',xmin=-5,xmax=3,ymin=-4,ymax=4);

\textgreater insimg;

\begin{figure}
\centering
\pandocbounded{\includegraphics[keepaspectratio]{images/Haifa Azka_23030530097 (APLIKOM pekan 11-12)-134.png}}
\caption{images/Haifa\%20Azka\_23030530097\%20(APLIKOM\%20pekan\%2011-12)-134.png}
\end{figure}

\textgreater fcontour(``d3'',xmin=-4,xmax=1,ymin=-2,ymax=2,hue=1,title=``The minimum is on A'');

\textgreater P=(A\_B\_C\_A)'; plot2d(P{[}1{]},P{[}2{]},add=1,color=12);

\textgreater insimg;

\begin{figure}
\centering
\pandocbounded{\includegraphics[keepaspectratio]{images/Haifa Azka_23030530097 (APLIKOM pekan 11-12)-135.png}}
\caption{images/Haifa\%20Azka\_23030530097\%20(APLIKOM\%20pekan\%2011-12)-135.png}
\end{figure}

\begin{enumerate}
\def\labelenumi{\arabic{enumi})}
\setcounter{enumi}{1}
\tightlist
\item
  Tetapi jika semua sudut segitiga ABC kurang dari 120°, maka titik minimum ada di titik F di bagian dalam segitiga, yaitu satu-satunya titik yang melihat sisi ABC dengan sudut yang sama (maka masing-masing sudutnya 120° ):
\end{enumerate}

\textgreater C={[}-0.5,1{]};

\textgreater plot3d(``d3'',xmin=-2,xmax=2,ymin=-2,ymax=2):

\begin{figure}
\centering
\pandocbounded{\includegraphics[keepaspectratio]{images/Haifa Azka_23030530097 (APLIKOM pekan 11-12)-136.png}}
\caption{images/Haifa\%20Azka\_23030530097\%20(APLIKOM\%20pekan\%2011-12)-136.png}
\end{figure}

\textgreater fcontour(``d3'',xmin=-2,xmax=2,ymin=-2,ymax=2,hue=1,title=``The Fermat point'');

\textgreater P=(A\_B\_C\_A)'; plot2d(P{[}1{]},P{[}2{]},add=1,color=12);

\textgreater insimg;

\begin{figure}
\centering
\pandocbounded{\includegraphics[keepaspectratio]{images/Haifa Azka_23030530097 (APLIKOM pekan 11-12)-137.png}}
\caption{images/Haifa\%20Azka\_23030530097\%20(APLIKOM\%20pekan\%2011-12)-137.png}
\end{figure}

Merupakan kegiatan yang menarik untuk merealisasikan gambar di atas dengan perangkat lunak geometri; misalnya, saya tahu soft tertulis di Java yang memiliki instruksi ``garis kontur''\ldots{}

Semua hal di atas ditemukan oleh seorang hakim Perancis bernama Pierrede Fermat; dia menulis surat kepada para penggila lainnya seperti pendeta Marin Mersenne dan Blaise Pascal yang bekerja di bagian pajak penghasilan. Jadi titik unik F sehingga FA+FB+FC minimal disebut titik Fermat segitiga. Namun nampaknya beberapa tahun sebelumnya, Torriccelli dari Italia telah menemukan titik ini sebelum Fermat menemukannya! Pokoknya tradisinya adalah memperhatikan hal ini F\ldots{}

\section{Empat poin}\label{empat-poin}

Langkah selanjutnya adalah menambahkan poin ke-4 D dan mencoba meminimalkan MA+MB+MC+MD; katakanlah Anda seorang operator TV kabel dan ingin mencari di bidang mana Anda harus memasang antena sehingga Anda dapat memberi makan empat desa dan menggunakan kabel sesedikit mungkin!

\textgreater D={[}1,1{]};

\textgreater function d4(x,y):=d3(x,y)+sqrt((x-D{[}1{]})\textsuperscript{2+(y-D{[}2{]})}2)

\textgreater plot3d(``d4'',xmin=-1.5,xmax=1.5,ymin=-1.5,ymax=1.5):

\begin{figure}
\centering
\pandocbounded{\includegraphics[keepaspectratio]{images/Haifa Azka_23030530097 (APLIKOM pekan 11-12)-138.png}}
\caption{images/Haifa\%20Azka\_23030530097\%20(APLIKOM\%20pekan\%2011-12)-138.png}
\end{figure}

\textgreater fcontour(``d4'',xmin=-1.5,xmax=1.5,ymin=-1.5,ymax=1.5,hue=1);

\textgreater P=(A\_B\_C\_D)'; plot2d(P{[}1{]},P{[}2{]},points=1,add=1,color=12);

\textgreater insimg;

\begin{figure}
\centering
\pandocbounded{\includegraphics[keepaspectratio]{images/Haifa Azka_23030530097 (APLIKOM pekan 11-12)-139.png}}
\caption{images/Haifa\%20Azka\_23030530097\%20(APLIKOM\%20pekan\%2011-12)-139.png}
\end{figure}

Masih ada nilai minimum dan tidak tercapai di simpul A, B, C, atau D:

\textgreater function f(x):=d4(x{[}1{]},x{[}2{]})

\textgreater neldermin(``f'',{[}0.2,0.2{]})

\begin{verbatim}
[0.142858,  0.142857]
\end{verbatim}

Nampaknya dalam hal ini koordinat titik optimal bersifat rasional atau mendekati rasional\ldots{}

Sekarang ABCD adalah persegi, kita berharap titik optimalnya adalah pusat ABCD:

\textgreater C={[}-1,1{]};

\textgreater plot3d(``d4'',xmin=-1,xmax=1,ymin=-1,ymax=1):

\begin{figure}
\centering
\pandocbounded{\includegraphics[keepaspectratio]{images/Haifa Azka_23030530097 (APLIKOM pekan 11-12)-140.png}}
\caption{images/Haifa\%20Azka\_23030530097\%20(APLIKOM\%20pekan\%2011-12)-140.png}
\end{figure}

\textgreater fcontour(``d4'',xmin=-1.5,xmax=1.5,ymin=-1.5,ymax=1.5,hue=1);

\textgreater P=(A\_B\_C\_D)'; plot2d(P{[}1{]},P{[}2{]},add=1,color=12,points=1);

\textgreater insimg;

\begin{figure}
\centering
\pandocbounded{\includegraphics[keepaspectratio]{images/Haifa Azka_23030530097 (APLIKOM pekan 11-12)-141.png}}
\caption{images/Haifa\%20Azka\_23030530097\%20(APLIKOM\%20pekan\%2011-12)-141.png}
\end{figure}

*Contoh 7: Bola Dandelin dengan Povray

Anda dapat menjalankan demonstrasi ini, jika Anda telah menginstal Povray, dan pvengine.exe di jalur program.

Pertama kita hitung jari-jari bola.

Jika diperhatikan gambar di bawah, terlihat bahwa kita membutuhkan dua lingkaran yang menyentuh dua garis yang membentuk kerucut, dan satu garis yang membentuk bidang yang memotong kerucut.

Kami menggunakan file geometri.e Euler untuk ini.

\textgreater load geometry;

First the two lines forming the cone.

\textgreater g1 \&= lineThrough({[}0,0{]},{[}1,a{]})

\begin{verbatim}
                             [- a, 1, 0]
\end{verbatim}

\textgreater g2 \&= lineThrough({[}0,0{]},{[}-1,a{]})

\begin{verbatim}
                            [- a, - 1, 0]
\end{verbatim}

Kemudian saya baris ketiga.

\textgreater g \&= lineThrough({[}-1,0{]},{[}1,1{]})

\begin{verbatim}
                             [- 1, 2, 1]
\end{verbatim}

Kita mem-plot semuanya sejauh ini.

\textgreater setPlotRange(-1,1,0,2);

\textgreater color(black); plotLine(g(),``\,``)

\textgreater a:=2; color(blue); plotLine(g1(),``\,``), plotLine(g2(),''\,``):

\begin{figure}
\centering
\pandocbounded{\includegraphics[keepaspectratio]{images/Haifa Azka_23030530097 (APLIKOM pekan 11-12)-142.png}}
\caption{images/Haifa\%20Azka\_23030530097\%20(APLIKOM\%20pekan\%2011-12)-142.png}
\end{figure}

Sekarang kita ambil titik umum pada sumbu y.

\textgreater P \&= {[}0,u{]}

\begin{verbatim}
                                [0, u]
\end{verbatim}

Hitung jarak ke g1.

\textgreater d1 \&= distance(P,projectToLine(P,g1)); \$d1

\[\sqrt{\left(\frac{a^2\,u}{a^2+1}-u\right)^2+\frac{a^2\,u^2}{\left(a  ^2+1\right)^2}}\]Hitung jarak ke g.

\textgreater d \&= distance(P,projectToLine(P,g)); \$d

\[\sqrt{\left(\frac{u+2}{5}-u\right)^2+\frac{\left(2\,u-1\right)^2}{  25}}\]Dan tentukan pusat kedua lingkaran yang jaraknya sama.

\textgreater sol \&= solve(d1\textsuperscript{2=d}2,u); \$sol

\[\left[ u=\frac{-\sqrt{5}\,\sqrt{a^2+1}+2\,a^2+2}{4\,a^2-1} , u=  \frac{\sqrt{5}\,\sqrt{a^2+1}+2\,a^2+2}{4\,a^2-1} \right] \]Ada dua solusi.

Kami mengevaluasi solusi simbolis, dan menemukan kedua pusat, dan kedua jarak.

\textgreater u := sol()

\begin{verbatim}
[0.333333,  1]
\end{verbatim}

\textgreater dd := d()

\begin{verbatim}
[0.149071,  0.447214]
\end{verbatim}

Plot lingkaran ke dalam gambar.

\textgreater color(red);

\textgreater plotCircle(circleWithCenter({[}0,u{[}1{]}{]},dd{[}1{]}),``\,``);

\textgreater plotCircle(circleWithCenter({[}0,u{[}2{]}{]},dd{[}2{]}),``\,``);

\textgreater insimg;

\begin{figure}
\centering
\pandocbounded{\includegraphics[keepaspectratio]{images/Haifa Azka_23030530097 (APLIKOM pekan 11-12)-146.png}}
\caption{images/Haifa\%20Azka\_23030530097\%20(APLIKOM\%20pekan\%2011-12)-146.png}
\end{figure}

\section{Plot dengan Povray}\label{plot-dengan-povray}

Selanjutnya kita plot semuanya dengan Povray. Perhatikan bahwa Anda mengubah perintah apa pun dalam urutan perintah Povray berikut, dan menjalankan kembali semua perintah dengan Shift-Return.

Pertama kita memuat fungsi povray.

\textgreater load povray;

\textgreater defaultpovray=``C:\textbackslash Program Files\textbackslash POV-Ray\textbackslash v3.7\textbackslash bin\textbackslash pvengine.exe''

\begin{verbatim}
C:\Program Files\POV-Ray\v3.7\bin\pvengine.exe
\end{verbatim}

Kami mengatur adegan dengan tepat.

\textgreater povstart(zoom=11,center={[}0,0,0.5{]},height=10°,angle=140°);

Selanjutnya kita menulis kedua bola tersebut ke file Povray.

\textgreater writeln(povsphere({[}0,0,u{[}1{]}{]},dd{[}1{]},povlook(red)));

\textgreater writeln(povsphere({[}0,0,u{[}2{]}{]},dd{[}2{]},povlook(red)));

Dan kerucutnya, transparan.

\textgreater writeln(povcone({[}0,0,0{]},0,{[}0,0,a{]},1,povlook(lightgray,1)));

Kami menghasilkan bidang yang dibatasi pada kerucut.

\textgreater gp=g();

\textgreater pc=povcone({[}0,0,0{]},0,{[}0,0,a{]},1,``\,``);

\textgreater vp={[}gp{[}1{]},0,gp{[}2{]}{]}; dp=gp{[}3{]};

\textgreater writeln(povplane(vp,dp,povlook(blue,0.5),pc));

Sekarang kita buat dua titik pada lingkaran, dimana bola menyentuh kerucut.

\textgreater function turnz(v) := return

\textgreater P1=projectToLine({[}0,u{[}1{]}{]},g1()); P1=turnz({[}P1{[}1{]},0,P1{[}2{]}{]});

\textgreater writeln(povpoint(P1,povlook(yellow)));

\textgreater P2=projectToLine({[}0,u{[}2{]}{]},g1()); P2=turnz({[}P2{[}1{]},0,P2{[}2{]}{]});

\textgreater writeln(povpoint(P2,povlook(yellow)));

Lalu kita buat dua titik di mana bola menyentuh bidang. Ini adalah fokus elips.

\textgreater P3=projectToLine({[}0,u{[}1{]}{]},g()); P3={[}P3{[}1{]},0,P3{[}2{]}{]};

\textgreater writeln(povpoint(P3,povlook(yellow)));

\begin{verbatim}
sphere { &lt;-0.0666667,0,0.466667&gt;, 0.02 texture { pigment { color rgb &lt;0.941176,0.941176,0.392157&gt; }  } 
 finish { ambient 0.2 } 

}
\end{verbatim}

\textgreater P4=projectToLine({[}0,u{[}2{]}{]},g()); P4={[}P4{[}1{]},0,P4{[}2{]}{]};

\textgreater writeln(povpoint(P4,povlook(yellow)));

\begin{verbatim}
sphere { &lt;0.2,0,0.6&gt;, 0.02 texture { pigment { color rgb &lt;0.941176,0.941176,0.392157&gt; }  } 
 finish { ambient 0.2 } 

}
\end{verbatim}

Next we compute the intersection of P1P2 with the plane.

\textgreater t1=scalp(vp,P1)-dp; t2=scalp(vp,P2)-dp; P5=P1+t1/(t1-t2)*(P2-P1);

\textgreater writeln(povpoint(P5,povlook(yellow)));

\begin{verbatim}
sphere { &lt;0,0.25,0.5&gt;, 0.02 texture { pigment { color rgb &lt;0.941176,0.941176,0.392157&gt; }  } 
 finish { ambient 0.2 } 

}
\end{verbatim}

We connect the points with line segments.

\textgreater writeln(povsegment(P1,P2,povlook(yellow)));

\begin{verbatim}
cylinder { &lt;0,0.133333,0.266667&gt;, &lt;0,0.4,0.8&gt;, 0.01
 texture { pigment { color rgb &lt;0.941176,0.941176,0.392157&gt; }  } 
 finish { ambient 0.2 } 
 }
\end{verbatim}

\textgreater writeln(povsegment(P5,P3,povlook(yellow)));

\begin{verbatim}
cylinder { &lt;0,0.25,0.5&gt;, &lt;-0.0666667,0,0.466667&gt;, 0.01
 texture { pigment { color rgb &lt;0.941176,0.941176,0.392157&gt; }  } 
 finish { ambient 0.2 } 
 }
\end{verbatim}

\textgreater writeln(povsegment(P5,P4,povlook(yellow)));

\begin{verbatim}
cylinder { &lt;0,0.25,0.5&gt;, &lt;0.2,0,0.6&gt;, 0.01
 texture { pigment { color rgb &lt;0.941176,0.941176,0.392157&gt; }  } 
 finish { ambient 0.2 } 
 }
\end{verbatim}

Now we generate a gray band, where the spheres touch the cone.

\textgreater pcw=povcone({[}0,0,0{]},0,{[}0,0,a{]},1.01);

\textgreater pc1=povcylinder({[}0,0,P1{[}3{]}-defaultpointsize/2{]},{[}0,0,P1{[}3{]}+defaultpointsize/2{]},1);

\textgreater writeln(povintersection({[}pcw,pc1{]},povlook(gray)));

\begin{verbatim}
intersection { cone { &lt;0,0,0&gt;, 0 &lt;0,0,2&gt;, 1.01 texture { pigment { color rgb &lt;0.470588,0.470588,0.470588&gt; }  } 
 finish { ambient 0.2 } 
 }
cylinder { &lt;0,0,0.256667&gt;, &lt;0,0,0.276667&gt;, 1
 texture { pigment { color rgb &lt;0.470588,0.470588,0.470588&gt; }  } 
 finish { ambient 0.2 } 
 }
 texture { pigment { color rgb &lt;0.470588,0.470588,0.470588&gt; }  } 
 finish { ambient 0.2 } 
 }
\end{verbatim}

\textgreater pc2=povcylinder({[}0,0,P2{[}3{]}-defaultpointsize/2{]},{[}0,0,P2{[}3{]}+defaultpointsize/2{]},1);

\textgreater writeln(povintersection({[}pcw,pc2{]},povlook(gray)));

\begin{verbatim}
intersection { cone { &lt;0,0,0&gt;, 0 &lt;0,0,2&gt;, 1.01 texture { pigment { color rgb &lt;0.470588,0.470588,0.470588&gt; }  } 
 finish { ambient 0.2 } 
 }
cylinder { &lt;0,0,0.79&gt;, &lt;0,0,0.81&gt;, 1
 texture { pigment { color rgb &lt;0.470588,0.470588,0.470588&gt; }  } 
 finish { ambient 0.2 } 
 }
 texture { pigment { color rgb &lt;0.470588,0.470588,0.470588&gt; }  } 
 finish { ambient 0.2 } 
 }
\end{verbatim}

Start the Povray program.

\textgreater povend();

\begin{figure}
\centering
\pandocbounded{\includegraphics[keepaspectratio]{images/Haifa Azka_23030530097 (APLIKOM pekan 11-12)-147.png}}
\caption{images/Haifa\%20Azka\_23030530097\%20(APLIKOM\%20pekan\%2011-12)-147.png}
\end{figure}

Untuk mendapatkan Anaglyph ini kita perlu memasukkan semuanya ke dalam fungsi scene. Fungsi ini akan digunakan dua kali kemudian.

\textgreater function scene () \ldots{}

\begin{verbatim}
global a,u,dd,g,g1,defaultpointsize;
writeln(povsphere([0,0,u[1]],dd[1],povlook(red)));
writeln(povsphere([0,0,u[2]],dd[2],povlook(red)));
writeln(povcone([0,0,0],0,[0,0,a],1,povlook(lightgray,1)));
gp=g();
pc=povcone([0,0,0],0,[0,0,a],1,"");
vp=[gp[1],0,gp[2]]; dp=gp[3];
writeln(povplane(vp,dp,povlook(blue,0.5),pc));
P1=projectToLine([0,u[1]],g1()); P1=turnz([P1[1],0,P1[2]]);
writeln(povpoint(P1,povlook(yellow)));
P2=projectToLine([0,u[2]],g1()); P2=turnz([P2[1],0,P2[2]]);
writeln(povpoint(P2,povlook(yellow)));
P3=projectToLine([0,u[1]],g()); P3=[P3[1],0,P3[2]];
writeln(povpoint(P3,povlook(yellow)));
P4=projectToLine([0,u[2]],g()); P4=[P4[1],0,P4[2]];
writeln(povpoint(P4,povlook(yellow)));
t1=scalp(vp,P1)-dp; t2=scalp(vp,P2)-dp; P5=P1+t1/(t1-t2)*(P2-P1);
writeln(povpoint(P5,povlook(yellow)));
writeln(povsegment(P1,P2,povlook(yellow)));
writeln(povsegment(P5,P3,povlook(yellow)));
writeln(povsegment(P5,P4,povlook(yellow)));
pcw=povcone([0,0,0],0,[0,0,a],1.01);
pc1=povcylinder([0,0,P1[3]-defaultpointsize/2],[0,0,P1[3]+defaultpointsize/2],1);
writeln(povintersection([pcw,pc1],povlook(gray)));
pc2=povcylinder([0,0,P2[3]-defaultpointsize/2],[0,0,P2[3]+defaultpointsize/2],1);
writeln(povintersection([pcw,pc2],povlook(gray)));
endfunction
\end{verbatim}

Anda memerlukan kacamata merah/cyan untuk melihat efek berikut.

\textgreater povanaglyph(``scene'',zoom=11,center={[}0,0,0.5{]},height=10°,angle=140°);

\begin{figure}
\centering
\pandocbounded{\includegraphics[keepaspectratio]{images/Haifa Azka_23030530097 (APLIKOM pekan 11-12)-148.png}}
\caption{images/Haifa\%20Azka\_23030530097\%20(APLIKOM\%20pekan\%2011-12)-148.png}
\end{figure}

\chapter{Contoh 8 : Geometri Bumi}\label{contoh-8-geometri-bumi}

Di buku catatan ini, kami ingin melakukan beberapa perhitungan bola. Fungsi-fungsi tersebut terdapat dalam file ``spherical.e'' di folder contoh. Kita perlu memuat file itu terlebih dahulu.

\textgreater load ``spherical.e'';

Untuk memasukkan posisi geografis, kita menggunakan vektor dengan dua koordinat dalam radian (utara dan timur, nilai negatif untuk selatan dan barat). Berikut koordinat Kampus FMIPA UNY.

\textgreater FMIPA={[}rad(-7,-46.467),rad(110,23.05){]}

\begin{verbatim}
[-0.13569,  1.92657]
\end{verbatim}

Anda dapat mencetak posisi ini dengan sposprint (cetak posisi bulat).

\textgreater sposprint(FMIPA) // posisi garis lintang dan garis bujur FMIPA UNY

\begin{verbatim}
S 7°46.467' E 110°23.050'
\end{verbatim}

Mari kita tambahkan dua kota lagi, Solo dan Semarang.

\textgreater Solo={[}rad(-7,-34.333),rad(110,49.683){]}; Semarang={[}rad(-6,-59.05),rad(110,24.533){]};

\textgreater sposprint(Solo), sposprint(Semarang),

\begin{verbatim}
S 7°34.333' E 110°49.683'
S 6°59.050' E 110°24.533'
\end{verbatim}

Pertama kita menghitung vektor dari satu bola ke bola ideal lainnya. Vektor ini adalah {[}pos, jarak{]} dalam radian. Untuk menghitung jarak di bumi, kita kalikan dengan jari-jari bumi pada garis lintang 7°.

\textgreater br=svector(FMIPA,Solo); degprint(br{[}1{]}), br{[}2{]}*rearth(7°)-\textgreater km // perkiraan jarak FMIPA-Solo

\begin{verbatim}
65°20'26.60''
53.8945384608
\end{verbatim}

Ini adalah perkiraan yang bagus. Rutinitas berikut menggunakan perkiraan yang lebih baik. Pada jarak sedekat itu, hasilnya hampir sama.

\textgreater esdist(FMIPA,Semarang)-\textgreater'' km'' // perkiraan jarak FMIPA-Semarang

\begin{verbatim}
Commands must be separated by semicolon or comma!
Found:  // perkiraan jarak FMIPA-Semarang (character 32)
You can disable this in the Options menu.
Error in:
esdist(FMIPA,Semarang)-&gt;" km" // perkiraan jarak FMIPA-Semaran ...
                             ^
\end{verbatim}

Judulnya ada fungsinya, dengan mempertimbangkan bentuk bumi yang elips. Sekali lagi, kami mencetak dengan cara yang canggih.

\textgreater sdegprint(esdir(FMIPA,Solo))

\begin{verbatim}
     65.34°
\end{verbatim}

Sudut suatu segitiga melebihi 180° pada bola.

\textgreater asum=sangle(Solo,FMIPA,Semarang)+sangle(FMIPA,Solo,Semarang)+sangle(FMIPA,Semarang,Solo); degprint(asum)

\begin{verbatim}
180°0'10.77''
\end{verbatim}

Ini dapat digunakan untuk menghitung luas segitiga. Catatan: Untuk segitiga kecil, ini tidak akurat karena kesalahan pengurangan pada asum-pi.

\textgreater(asum-pi)*rearth(48°)\^{}2-\textgreater'' km\^{}2'' // perkiraan luas segitiga FMIPA-Solo-Semarang

\begin{verbatim}
Commands must be separated by semicolon or comma!
Found:  // perkiraan luas segitiga FMIPA-Solo-Semarang (character 32)
You can disable this in the Options menu.
Error in:
(asum-pi)*rearth(48°)^2-&gt;" km^2" // perkiraan luas segitiga FM ...
                                ^
\end{verbatim}

Ada fungsi untuk ini, yang menggunakan garis lintang rata-rata dari segitiga untuk menghitung jari-jari bumi, dan menangani kesalahan pembulatan untuk segitiga yang sangat kecil.

\textgreater esarea(Solo,FMIPA,Semarang)-\textgreater'' km\^{}2'', //perkiraan yang sama dengan fungsi esarea()

\begin{verbatim}
2123.64310526 km^2
\end{verbatim}

Kita juga dapat menambahkan vektor ke posisi. Vektor berisi arah dan jarak, keduanya dalam radian. Untuk mendapatkan vektor, kita menggunakan svector. Untuk menambahkan vektor ke suatu posisi, kita menggunakan saddvector.

\textgreater v=svector(FMIPA,Solo); sposprint(saddvector(FMIPA,v)), sposprint(Solo),

\begin{verbatim}
S 7°34.333' E 110°49.683'
S 7°34.333' E 110°49.683'
\end{verbatim}

Fungsi-fungsi ini mengasumsikan bola ideal. Hal yang sama terjadi di bumi.

\textgreater sposprint(esadd(FMIPA,esdir(FMIPA,Solo),esdist(FMIPA,Solo))), sposprint(Solo),

\begin{verbatim}
S 7°34.333' E 110°49.683'
S 7°34.333' E 110°49.683'
\end{verbatim}

Mari kita lihat contoh yang lebih besar, Tugu Jogja dan Monas Jakarta (menggunakan Google Earth untuk mencari koordinatnya).

\textgreater Tugu={[}-7.7833°,110.3661°{]}; Monas={[}-6.175°,106.811944°{]};

\textgreater sposprint(Tugu), sposprint(Monas)

\begin{verbatim}
S 7°46.998' E 110°21.966'
S 6°10.500' E 106°48.717'
\end{verbatim}

Menurut Google Earth, jaraknya 429,66km. Kami mendapatkan perkiraan yang bagus.

\textgreater esdist(Tugu,Monas)-\textgreater'' km'' // perkiraan jarak Tugu Jogja - Monas Jakarta

\begin{verbatim}
Commands must be separated by semicolon or comma!
Found:  // perkiraan jarak Tugu Jogja - Monas Jakarta (character 32)
You can disable this in the Options menu.
Error in:
esdist(Tugu,Monas)-&gt;" km" // perkiraan jarak Tugu Jogja - Mona ...
                         ^
\end{verbatim}

Judulnya sama dengan yang dihitung di Google Earth.

\textgreater degprint(esdir(Tugu,Monas))

\begin{verbatim}
294°17'2.85''
\end{verbatim}

Namun kita tidak lagi mendapatkan posisi sasaran yang tepat, jika kita menambahkan heading dan jarak ke posisi semula. Hal ini terjadi karena kita tidak menghitung fungsi invers secara tepat, namun melakukan perkiraan jari-jari bumi di sepanjang lintasan.

\textgreater sposprint(esadd(Tugu,esdir(Tugu,Monas),esdist(Tugu,Monas)))

\begin{verbatim}
S 6°10.500' E 106°48.717'
\end{verbatim}

Namun kesalahannya tidak besar.

\textgreater sposprint(Monas),

\begin{verbatim}
S 6°10.500' E 106°48.717'
\end{verbatim}

Tentu kita tidak bisa berlayar dengan tujuan yang sama dari satu tujuan ke tujuan lainnya, jika ingin mengambil jalur terpendek. Bayangkan, Anda terbang NE mulai dari titik mana saja di bumi. Kemudian Anda akan berputar ke kutub utara. Lingkaran besar tidak mengikuti arah yang konstan!

Perhitungan berikut menunjukkan bahwa kita jauh dari tujuan yang benar, jika kita menggunakan tujuan yang sama selama perjalanan.

\textgreater dist=esdist(Tugu,Monas); hd=esdir(Tugu,Monas);

Sekarang kita tambah 10 kali sepersepuluh jarak, pakai jurusan Monas, kita sampai di Tugu.

\textgreater p=Tugu; loop 1 to 10; p=esadd(p,hd,dist/10); end;

Hasilnya jauh sekali.

\textgreater sposprint(p), skmprint(esdist(p,Monas))

\begin{verbatim}
S 6°11.250' E 106°48.372'
     1.529km
\end{verbatim}

Sebagai contoh lain, mari kita ambil dua titik di bumi yang mempunyai garis lintang yang sama.

\textgreater P1={[}30°,10°{]}; P2={[}30°,50°{]};

Jalur terpendek dari P1 ke P2 bukanlah lingkaran dengan garis lintang 30°, melainkan jalur yang lebih pendek yang dimulai 10° lebih jauh ke utara di P1.

\textgreater sdegprint(esdir(P1,P2))

\begin{verbatim}
     79.69°
\end{verbatim}

Namun, jika kita mengikuti pembacaan kompas ini, kita akan berputar ke kutub utara! Jadi kita harus menyesuaikan arah perjalanan kita. Untuk tujuan kasarnya, kita sesuaikan pada 1/10 dari total jarak.

\textgreater p=P1; dist=esdist(P1,P2); \ldots{}\\
\textgreater{} loop 1 to 10; dir=esdir(p,P2); sdegprint(dir), p=esadd(p,dir,dist/10); end;

\begin{verbatim}
     79.69°
     81.67°
     83.71°
     85.78°
     87.89°
     90.00°
     92.12°
     94.22°
     96.29°
     98.33°
\end{verbatim}

Jaraknya tidak tepat, karena kita akan menambahkan sedikit kesalahan jika kita mengikuti arah yang sama terlalu lama.

\textgreater skmprint(esdist(p,P2))

\begin{verbatim}
     0.203km
\end{verbatim}

Kita mendapatkan perkiraan yang baik, jika kita menyesuaikan arah setiap 1/100 dari total jarak dari Tugu ke Monas.

\textgreater p=Tugu; dist=esdist(Tugu,Monas); \ldots{}\\
\textgreater{} loop 1 to 100; p=esadd(p,esdir(p,Monas),dist/100); end;

\textgreater skmprint(esdist(p,Monas))

\begin{verbatim}
     0.000km
\end{verbatim}

Untuk keperluan navigasi, kita bisa mendapatkan urutan posisi GPS sepanjang lingkaran besar menuju Monas dengan fungsi navigasi.

\textgreater load spherical; v=navigate(Tugu,Monas,10); \ldots{}\\
\textgreater{} loop 1 to rows(v); sposprint(v{[}\#{]}), end;

\begin{verbatim}
S 7°46.998' E 110°21.966'
S 7°37.422' E 110°0.573'
S 7°27.829' E 109°39.196'
S 7°18.219' E 109°17.834'
S 7°8.592' E 108°56.488'
S 6°58.948' E 108°35.157'
S 6°49.289' E 108°13.841'
S 6°39.614' E 107°52.539'
S 6°29.924' E 107°31.251'
S 6°20.219' E 107°9.977'
S 6°10.500' E 106°48.717'
\end{verbatim}

Kita menulis sebuah fungsi yang memplot bumi, dua posisi, dan posisi di antaranya.

\textgreater function testplot \ldots{}

\begin{verbatim}
useglobal;
plotearth;
plotpos(Tugu,"Tugu Jogja"); plotpos(Monas,"Tugu Monas");
plotposline(v);
endfunction
\end{verbatim}

Sekarang Plot semuanya

\textgreater plot3d(``testplot'',angle=25, height=6,\textgreater own,\textgreater user,zoom=4):

\begin{figure}
\centering
\pandocbounded{\includegraphics[keepaspectratio]{images/Haifa Azka_23030530097 (APLIKOM pekan 11-12)-149.png}}
\caption{images/Haifa\%20Azka\_23030530097\%20(APLIKOM\%20pekan\%2011-12)-149.png}
\end{figure}

Atau gunakan plot3d untuk mendapatkan tampilan anaglyph. Ini terlihat sangat bagus dengan kacamata merah/cyan.

\textgreater plot3d(``testplot'',angle=25,height=6,distance=5,own=1,anaglyph=1,zoom=4):

\begin{figure}
\centering
\pandocbounded{\includegraphics[keepaspectratio]{images/Haifa Azka_23030530097 (APLIKOM pekan 11-12)-150.png}}
\caption{images/Haifa\%20Azka\_23030530097\%20(APLIKOM\%20pekan\%2011-12)-150.png}
\end{figure}

\textgreater load geometry

\begin{verbatim}
Numerical and symbolic geometry.
\end{verbatim}

\chapter{Latihan}\label{latihan-1}

\begin{enumerate}
\def\labelenumi{\arabic{enumi}.}
\tightlist
\item
  Gambarlah segi-n beraturan jika diketahui titik pusat O, n, dan jarak titik pusat ke titik-titik sudut segi-n tersebut (jari-jari lingkaran luar segi-n), r.
\end{enumerate}

Petunjuk:

\begin{itemize}
\item
  Besar sudut pusat yang menghadap masing-masing sisi segi-n adalah
\item
  (360/n).
\item
  Titik-titik sudut segi-n merupakan perpotongan lingkaran luar segi-n
\item
  dan garis-garis yang melalui pusat dan saling membentuk sudut sebesar
\item
  kelipatan (360/n).
\item
  Untuk n ganjil, pilih salah satu titik sudut adalah di atas.
\item
  Untuk n genap, pilih 2 titik di kanan dan kiri lurus dengan titik
\item
  pusat.
\item
  Anda dapat menggambar segi-3, 4, 5, 6, 7, dst beraturan.
\end{itemize}

\textgreater setPlotRange(-3.3,3.3,-3.3,3.3);

\textgreater O={[}0,0{]}; plotPoint(O,``O'');

\textgreater A={[}-3,-2{]}; plotPoint(A,``A'')

\textgreater B={[}2,-2{]}; plotPoint(B,``B'');

\textgreater C={[}0,2*3\^{}0.5-1{]}; plotPoint(A,``A'');

\textgreater plotSegment(A,B,``c'');

\textgreater plotSegment(B,C,``a'');

\textgreater plotSegment(A,C,``b'');

\textgreater aspect(1):

\begin{figure}
\centering
\pandocbounded{\includegraphics[keepaspectratio]{images/Haifa Azka_23030530097 (APLIKOM pekan 11-12)-151.png}}
\caption{images/Haifa\%20Azka\_23030530097\%20(APLIKOM\%20pekan\%2011-12)-151.png}
\end{figure}

\textgreater c=circleThrough(A,B,C):

\begin{figure}
\centering
\pandocbounded{\includegraphics[keepaspectratio]{images/Haifa Azka_23030530097 (APLIKOM pekan 11-12)-152.png}}
\caption{images/Haifa\%20Azka\_23030530097\%20(APLIKOM\%20pekan\%2011-12)-152.png}
\end{figure}

\textgreater R=getCircleRadius(c);

\textgreater O=getCircleCenter(c)

\begin{verbatim}
[-0.5,  -0.439977]
\end{verbatim}

\textgreater plotCircle(c,``Lingkaran luar segitiga''):

\begin{figure}
\centering
\pandocbounded{\includegraphics[keepaspectratio]{images/Haifa Azka_23030530097 (APLIKOM pekan 11-12)-153.png}}
\caption{images/Haifa\%20Azka\_23030530097\%20(APLIKOM\%20pekan\%2011-12)-153.png}
\end{figure}

\begin{enumerate}
\def\labelenumi{\arabic{enumi}.}
\setcounter{enumi}{1}
\tightlist
\item
  Gambarlah suatu parabola yang melalui 3 titik yang diketahui.
\end{enumerate}

Petunjuk:

\begin{itemize}
\item
  Misalkan persamaan parabolanya y= ax\^{}2+bx+c.
\item
  Substitusikan koordinat titik-titik yang diketahui ke persamaan tersebut.
\item
  Selesaikan SPL yang terbentuk untuk mendapatkan nilai-nilai a, b, c.
\end{itemize}

\textgreater setPlotRange(10);

\textgreater G={[}-6,0{]}; H={[}6,0{]}; I={[}0,9{]};

\textgreater plotPoint(G,``G'');

\textgreater plotPoint(H,``H'');

\textgreater plotPoint(I,``I''):

\begin{figure}
\centering
\pandocbounded{\includegraphics[keepaspectratio]{images/Haifa Azka_23030530097 (APLIKOM pekan 11-12)-154.png}}
\caption{images/Haifa\%20Azka\_23030530097\%20(APLIKOM\%20pekan\%2011-12)-154.png}
\end{figure}

\textgreater sol \&= solve({[}16*a+8*b=-c, 16*a-8*b=-c,c=2{]},{[}a,b,c{]})

\begin{verbatim}
                                  []
\end{verbatim}

\textgreater function y\&=-1/8*(x\^{}2)-0*x+2

\begin{verbatim}
                                     2
                                    x
                                2 - --
                                    8
\end{verbatim}

\textgreater plot2d(``-1/8*(x\^{}2)-0*x+2'',-6,6);

\textgreater plotPoint(G,``G''); plotPoint(H,``H''); plotPoint(I,``I''):

\begin{figure}
\centering
\pandocbounded{\includegraphics[keepaspectratio]{images/Haifa Azka_23030530097 (APLIKOM pekan 11-12)-155.png}}
\caption{images/Haifa\%20Azka\_23030530097\%20(APLIKOM\%20pekan\%2011-12)-155.png}
\end{figure}

\begin{enumerate}
\def\labelenumi{\arabic{enumi}.}
\setcounter{enumi}{2}
\item
  Gambarlah suatu segi-4 yang diketahui keempat titik sudutnya, misalnya A, B, C, D.

  \begin{itemize}
  \item
    Tentukan apakah segi-4 tersebut merupakan segi-4 garis singgung (sisinya-sisintya merupakan garis singgung lingkaran yang sama yakni lingkaran dalam segi-4 tersebut).
  \item
    Suatu segi-4 merupakan segi-4 garis singgung apabila keempat garis bagi sudutnya bertemu di satu titik.
  \item
    Jika segi-4 tersebut merupakan segi-4 garis singgung, gambar lingkaran dalamnya.
  \item
    Tunjukkan bahwa syarat suatu segi-4 merupakan segi-4 garis singgung apabila hasil kali panjang sisi-sisi yang berhadapan sama.
  \end{itemize}
\end{enumerate}

\textgreater setPlotRange(10)

\begin{verbatim}
[-10,  10,  -10,  10]
\end{verbatim}

\textgreater A={[}-6,-6{]}; B={[}6,-6{]}; C={[}6,-6{]}; D={[}-6,6{]};

\textgreater plotPoint(A,``A''); plotPoint(B,``B''); plotPoint(C,``C''); plotPoint(D,``D'');

\textgreater plotSegment(A,B,``\,``); plotSegment(B,C,''``); plotSegment(C,D,''``); plotSegment(D,A,''\,``):

\begin{figure}
\centering
\pandocbounded{\includegraphics[keepaspectratio]{images/Haifa Azka_23030530097 (APLIKOM pekan 11-12)-156.png}}
\caption{images/Haifa\%20Azka\_23030530097\%20(APLIKOM\%20pekan\%2011-12)-156.png}
\end{figure}

\textgreater l=angleBisector(A,B,C);

\begin{verbatim}
Function angleBisector needs a vector for A
Error in:
l=angleBisector(A,B,C); ...
                      ^
\end{verbatim}

\textgreater g=angleBisector(B,C,D);

\begin{verbatim}
Function angleBisector needs a vector for A
Error in:
g=angleBisector(B,C,D); ...
                      ^
\end{verbatim}

\textgreater P=lineIntersection(l,g)

\begin{verbatim}
Function lineIntersection needs a vector for h
Error in:
P=lineIntersection(l,g) ...
                       ^
\end{verbatim}

\textgreater color(5); plotLine(l); plotLine(g); color(1);

\begin{verbatim}
Function plotLine needs a vector for g
Error in:
color(5); plotLine(l); plotLine(g); color(1); ...
                                  ^
\end{verbatim}

\textgreater r=norm(P-projectToLine(P,lineThrough(A,B)))

\begin{verbatim}
Function lineThrough needs a vector for A
Error in:
r=norm(P-projectToLine(P,lineThrough(A,B))) ...
                                         ^
\end{verbatim}

\textgreater plotCircle(circleWithCenter(P,r),``Lingkaran dalam segi4 ABC''):

\begin{verbatim}
Function circleWithCenter needs a vector for A
Error in:
plotCircle(circleWithCenter(P,r),"Lingkaran dalam segi4 ABC"): ...
                                ^
\end{verbatim}

\textgreater AB=norm(A-B)

\begin{verbatim}
12
\end{verbatim}

\textgreater BC=norm(B-C)

\begin{verbatim}
0
\end{verbatim}

\textgreater CD=norm(C-D)

\begin{verbatim}
16.9705627485
\end{verbatim}

\textgreater DA=norm(D-A)

\begin{verbatim}
12
\end{verbatim}

\textgreater AB.CD

\begin{verbatim}
203.646752982
\end{verbatim}

\textgreater DA.BC

\begin{verbatim}
0
\end{verbatim}

\begin{enumerate}
\def\labelenumi{\arabic{enumi}.}
\setcounter{enumi}{3}
\tightlist
\item
  Gambarlah suatu ellips jika diketahui kedua titik fokusnya, misalnya P dan Q. Ingat ellips dengan fokus P dan Q adalah tempat kedudukan titik-titik yang jumlah jarak ke P dan ke Q selalu sama (konstan).
\end{enumerate}

\textgreater P={[}-3,5{]}; Q={[}3,5{]}

\begin{verbatim}
[3,  5]
\end{verbatim}

\textgreater function d1(x,y):=sqrt((x-P{[}1{]})\textsuperscript{2+(y-P{[}2{]})}2)

\textgreater function d2(x,y):=d1(x,y)+sqrt((x-Q{[}1{]})\textsuperscript{2+(y-Q{[}2{]})}2)

\textgreater fcontour(``d2'',xmin=-2,xmax=-2,ymin=0,ymax=4,hue=1):

\begin{figure}
\centering
\pandocbounded{\includegraphics[keepaspectratio]{images/Haifa Azka_23030530097 (APLIKOM pekan 11-12)-157.png}}
\caption{images/Haifa\%20Azka\_23030530097\%20(APLIKOM\%20pekan\%2011-12)-157.png}
\end{figure}

\textgreater plot3d(``d2'',xmin=-2,xmax=2,ymin=0,ymax=4,hue=1):

\begin{figure}
\centering
\pandocbounded{\includegraphics[keepaspectratio]{images/Haifa Azka_23030530097 (APLIKOM pekan 11-12)-158.png}}
\caption{images/Haifa\%20Azka\_23030530097\%20(APLIKOM\%20pekan\%2011-12)-158.png}
\end{figure}

\textgreater plot2d(``abs(x+1)+abs(x-1)'',xmin=-3,xmax=3):

\begin{figure}
\centering
\pandocbounded{\includegraphics[keepaspectratio]{images/Haifa Azka_23030530097 (APLIKOM pekan 11-12)-159.png}}
\caption{images/Haifa\%20Azka\_23030530097\%20(APLIKOM\%20pekan\%2011-12)-159.png}
\end{figure}

\begin{enumerate}
\def\labelenumi{\arabic{enumi}.}
\setcounter{enumi}{4}
\tightlist
\item
  Gambarlah suatu hiperbola jika diketahui kedua titik fokusnya, misalnya P dan Q. Ingat ellips dengan fokus P dan Q adalah tempat kedudukan titik-titik yang selisih jarak ke P dan ke Q selalu sama (konstan).
\end{enumerate}

\backmatter
\end{document}
